
% Resumo
%
% Escolha de 3 a 6 palavras ou termos que descrevam bem o seu trabalho. As
% palavras-chaves são utilizadas para indexação. A letra inicial de cada palavra
% deve estar em maiúsculas. As palavras-chave são separadas por ponto.
%

\begin{resumo}
    Atividades de desenvolvimento, manutenção e evolução de software apresentam
    elevada complexidade, uma vez que questões pertinentes à sua prática vão de
    técnicas e tecnologias a questões socioculturais.
	Para que se possa alterar corretamente um sistema de software, é necessário
	compreender o domínio de aplicação, os aspectos técnicos e tecnológicos
	envolvidos, bem como o próprio sistema em questão.
	Nesse sentido, propomos um modelo para análise de software em execução
	através de redes hierárquicas, heterogêneas e navegáveis em múltiplas
	escalas, com ligações e componentes topológicos.
	Como certas propriedades dos sistemas de software são conhecidas somente
	durante sua execução, o modelo é, em sua maior parte, construído a partir de
	trilhas de execução reduzidas no espaço e no tempo.
	Técnicas de análise estática podem ser usadas para complementá-lo.
	O resultado da aplicação do modelo nos sistemas sob análise é um mapa da
	estrutura de suas execuções.
	Realizamos estudo de caso na aplicação \texttt{gulp}, um
	\textit{taks runner} Javascript.
	Resultados preliminares indicam necessidade de analisar software como um
	sistema de sistemas.

    \textbf{Palavras-chave}:
	Redes Complexas de Software.
	Redes Complexas Hierárquicas.
	Renderização de grafos orientada a forças.
    Análise Híbrida de Software.
	Visualização de Software. 
\end{resumo}
