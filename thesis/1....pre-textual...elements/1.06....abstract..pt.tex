
% Resumo
%
% Escolha de 3 a 6 palavras ou termos que descrevam bem o seu trabalho. As
% palavras-chaves são utilizadas para indexação. A letra inicial de cada palavra
% deve estar em maiúsculas. As palavras-chave são separadas por ponto.
%

\begin{resumo}
    Atividades de desenvolvimento, manutenção e evolução de software apresentam
    elevada complexidade, uma vez que questões pertinentes à sua prática vão de
    técnicas e tecnologias a questões socioculturais. Para que se possa alterar
    corretamente um sistema de software, é necessário não somente compreender os
    aspectos técnicos e tecnológicos envolvidos, mas também o próprio sistema em
    questão. Nesse sentido, propomos um modelo para análise de software em
    execução através de redes hierárquicas, heterogêneas e navegáveis em
    múltiplas escalas, com ligações e componentes topológicos. Como certas
    propriedades de sistemas de software são conhecidas somente durante sua
    execução, o modelo é construído a partir de trilhas instrumentadas,
    reduzidas no espaço e no tempo. O resultado da aplicação de tal modelo é um
    mapa da estrutura de execução de um sistema de software. Este trabalho será
    avaliado através de estudos de casos.

    \textbf{Palavras-chave}: Redes Complexas Hierárquicas.
    Análise Dinâmica de Software. Visualização de Software. Modelos de Software.
    Engenharia de Software Colaborativa.
\end{resumo}
