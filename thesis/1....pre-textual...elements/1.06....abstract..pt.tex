
% Resumo
%
% Escolha de 3 a 6 palavras ou termos que descrevam bem o seu trabalho. As
% palavras-chaves são utilizadas para indexação. A letra inicial de cada palavra
% deve estar em maiúsculas. As palavras-chave são separadas por ponto.
%

\begin{resumo}
    Atividades de desenvolvimento, manutenção e evolução de software apresentam
    elevada complexidade, uma vez que questões pertinentes à sua prática vão de
    técnicas e tecnologias a questões socioculturais. Certas propriedades de
    sistemas de software somente são conhecidas durante sua execução. Poucos são
    os trabalhos que tratam da análise de sua estrutura dinâmica; no entanto. O
    presente trabalho propõe um modelo que aplica união e redução em trilhas de
    execução e gera uma rede de componentes interconectados e colaborativos. É
    esperado que tais redes representem com fidelidade a estrutura do software
    em execução. Serão discutidas aplicações e limitações, possível
    implementação, dificuldades técnicas e tecnológicas, bem como possibilidades
    colaborativas na Engenharia de Software de tal modelo.

    \textbf{Palavras-chave}: Redes Complexas Hierárquicas.
    Análise Dinâmica de Software. Visualização de Software. Modelos de Software.
    Engenharia de Software Colaborativa.
\end{resumo}
