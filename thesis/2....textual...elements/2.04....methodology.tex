
% Metodologia
%

\chapter{Metodologia}
\label{Chapter:Methodology}

A metodologia deste trabalho consiste na proposição de um modelo da estrutura da
execução de sistemas de software através de redes hierárquicas, heterogêneas e
navegáveis em múltiplas escalas, com ligações e componentes topológicos.
Futuramente, este capítulo conterá detalhes da implementação de ferramentas para
a coleta das trilhas de execução e suas transfomações, bem como para a
construção da rede, suas operações e renderização.

Exemplos da aplicação do modelo a casos didáticos constam no Apêndice
\ref{Chapter:SpatialTemporalOperators}. Conduzimos estudo de caso no qual
avaliamos o sistema \texttt{gulp}, um \textit{task runner} JavaScript, no
Capítulo \ref{Chapter:Results}.

\section{Modelo Proposto}
\label{sec:PropusedModel}

O modelo proposto visa representar a estrutura de execução de sistemas de
software através de uma rede, onde os elementos são componentes desses sistemas
e ligações são relações entre tais componentes.

Os conceitos de componente e ligação aqui definidos são abstratos, uma vez que a
proposta é agnóstica às linguagens de programação.
Se tratarmos de sistemas orientados a objetos, os compontes elementares da rede
serão métodos; se avaliarmos sistemas funcionais, funções.
Outros cenários também são compatíveis com essa abordagem: em JavaScript, por
exemplo, funções são consideradas métodos, caso definidas como propriedades de
um objeto. Neste caso um componente elementar ora é função, ora é método. 

As ligações entre componentes elementares representam relações direcionadas
onde, dado um componente originário $a$ e um componente destinatário $b$,
estabelecemos que $a$ executa $b$.
Utilizaremos o símbolo $\mathcal{N}$ para denotar redes formadas apenas por
componentes elementares e suas ligações.

O modelo prevê a inserção de componentes e ligações topológicas, de modo que as
redes produzidas se tornam heterogêneas e hierárquicas.
Isto ocorre porque as redes passam a conter diferentes tipos de componentes, com
diferentes relações hierárquicas entre si.
São exemplos de componentes topológicos arquivos, pastas, pacotes e outros
elementos estruturais obtidos a partir de análise estática ou dinâmica.
Dado um componente originário de maior nível hierárquico $a$ e um componente
destinatário de menor nível hierárquico $b$, as relações entre $a$ e $b$ podem
ser caracterizadas por palavras como contém, define, usa, entre outros.
Redes que contenham componentes elementares e topológicos e suas ligações serão
denotadas pelo símbolo $\mathcal{M}$.

\subsection{Construção da Rede}
\label{subsec:DataAquisition}

A construção de $\mathcal{N}$ se dá pela redução espacial e temporal de trilhas
de execução. $\mathcal{M}$ é obtido através da inserção de componentes
topológicos em $\mathcal{N}$.
Se considerarmos um componente como uma localidade da estrutura de execução do
sistema, podemos definir o operador de redução espacial --- denotado por
$\mathcal{S}$ --- como uma transformação em uma trilha de execução que unifica
diferentes ocorrências de um mesmo componente.
Se considerarmos um conjunto qualquer de trilhas de execução, o operador de
redução temporal --- denotado por $\mathcal{T}$ --- é a união dessas
trilhas, reduzidas espacialmente ou não, em uma única estrutura.
Observe que $\mathcal{S}$ atua sobre uma trilha de execução, enquanto que
$\mathcal{T}$ atua sobre um conjunto de trilhas.

Seja $X = \{ \ x_1, \ x_2, \ x_3, \ \dots, \ x_n \ \}$ um conjunto de $n$
trilhas de execução, representados por árvores.
Seja $\dot{X} = \{ \ \dot{x_1}, \ \dot{x_2}, \ \dot{x_3}, \ \dots, \ \dot{x_n} \ \}$
como conjunto de $n$ trilhas de execução reduzidas espacialmente, representados
por grafos.
A relação entre $x_i$ e $\dot{x_i}$ se dá por

\begin{equation}
	\dot{x_i} = \mathcal{S}(x_i),
\end{equation}

\noindent
e entre $X$ e $\dot{X}$ por

\begin{equation}
	\dot{X} = \mathcal{S}(X).
\end{equation}

Mapa $\mathcal{N}$ da estrutura de execução de determinado sistema de software
pode ser obtido por

\begin{equation}
	\mathcal{N} = \mathcal{T}(\mathcal{S}(X)) = \bigcup\limits_{i=1}^{n} \mathcal{S}(x_i).
\end{equation}

Os pseudo-códigos referentes aos operadores $\mathcal{S}$ e $\mathcal{T}$
constam nos Algoritmos \hyperref[Algorithm:SpatialReduction]{3.1} e
\hyperref[Algorithm:TemporalReduction]{3.2}, respectivamente. A expressão
\texttt{para cada} no Algoritmo \hyperref[Algorithm:SpatialReduction]{3.2}
corresponde a percorrer todos os nós de uma árvore que representa uma trilha de
execução.

\begin{algorithm}
    \DontPrintSemicolon 

    \label{Algorithm:SpatialReduction}
    \caption{Operador $\mathcal{S}$ de redução espacial sobre trilha de execução}
    \BlankLine

    \Entrada{trilha de execução $\dot{x}$, trilha de execução reduzida espacialmente $x$}
    \Saida{trilha de execução reduzida espacialmente $x$}
    \BlankLine

    \SetKwFunction{FSpatialReduction}{$\mathcal{S}$}
    \SetKwProg{Operator}{Operador}{:}{\Retorna $x$}
    \Operator{\FSpatialReduction{$\dot{x}, x$}}{
        \BlankLine

        \ParaCada {nó $n\;|\; n \in \dot{x}$} {
            seja $f$ nó antecedente de $n$\\

            \BlankLine
            \Se {$n \notin x$} {
                adiciona vértice $n$ ao grafo $x$\\
            }

            \BlankLine
            \Se {aresta $e(f, n) \notin x$ e $f$ não é nulo } {
                adiciona aresta $e(f, n)$ ao grafo $x$\\
            }

            \BlankLine
            \Se {$f$ é nulo } {
                marca $n$ como ponto de entrada\\
            }
        }
        \BlankLine

    }
\end{algorithm}


\begin{algorithm}
    \DontPrintSemicolon

    \label{Algorithm:TemporalReduction}
    \caption{Operador $\mathcal{T}$ de redução temporal sobre trilhas de execução}
    \BlankLine

    \Entrada{conjunto de trilhas de execução $\dot{X}$}
    \Saida{mapa $\mathcal{M}$ da estrutura de execução}
    \BlankLine

    \SetKwFunction{FTemporalReduction}{$\mathcal{T}$}
    \SetKwProg{Operator}{Operador}{:}{\Retorna $\mathcal{M}$}
    \Operator{\FTemporalReduction{$\dot{X}$}}{
        $x \leftarrow$ grafo vazio\\

        \BlankLine
        \ParaCada {trilha de execução $\dot{x}\;|\;\dot{x} \in \dot{X}$} {
            $x \leftarrow \mathcal{S}(\dot{x}, x)$\\
        }
        \BlankLine

        $\mathcal{M} \leftarrow x$
        \BlankLine
    }
\end{algorithm}


Neste ponto basta inserir componentes topológicos em $\mathcal{N}$ para que se
obtenha $\mathcal{M}$.

\subsection{Valores de Componentes e suas Ligações}
\label{subsec:ModelCharacterization}

Atribuímos tipicamente aos componentes elementares o índice $i_e = 1$.
Para componetes topológicos, quão maior o nível de abstração, maior seu índice.
Uma opção é utilizar a sequência ordinal para $n$ classes hierárquicas, onde o
índice de um componente qualquer é o $k$-\textit{ésimo} elemento do conjunto
$\{\ 1,\ 2,\ \dots,\ n\ \}$ de acordo com sua classe.
Determinaremos o índice de um componente topológico como a soma dos índices dos
componentes em que ele se conecta.

Os valores para os componentes e ligações da rede são dependentes de uma função
indexadora $f(i)$ e série numérica $N$, associadas a uma escala $S$

\begin{equation}
	\label{Equation:SetScale}
	S(N, f(i)) = N_{f(i)}.
\end{equation}

A escala linear --- ou natural --- é definda pelo conjunto dos números naturais
maiores que zero

\begin{equation}
	\label{Equation:NaturalSetScale}
	\mathbb{N}_{i}^{*}.
\end{equation}

A escala de Fibonacci é definida como

\begin{equation}
	\label{Equation:FibonacciSetScale}
	F_{i+2}, \qquad F= \{\ 0,\ 1,\ 1,\ 2,\ 3,\ 5,\ 8,\ \dots\ \}.
\end{equation}

Note que a função de índice expressa pela Equação \ref{Equation:FibonacciSetScale}
nos permite desconsiderar os dois primeiros elementos da série de Fibonacci
--- o elemento 0 e o primeiro elemento 1 --- para que não se represente um
componete com valor 0 e componentes em diferentes níveis hierárquicos com o
mesmo valor 1.

O valor $v$ de um componente é

\begin{equation}
	\label{Equation:WeightVertex1}
	v = \alpha_1 \cdot N_{f(i)},
\end{equation}

\noindent
onde \boldsymbol{$\alpha$} é um vetor de parâmetros de ajuste.

Dado um componente originário $a$ e um componente destinatário $b$, com os
respectivos índices $i_a$ e $i_b$, o valor $e$ de uma ligação é determinada por  

\begin{equation}
	\label{Equation:WeightEdge1}
	e = \frac{1}{\alpha_2 \cdot N_{f(i_a)}},
\end{equation}

\noindent
de modo que quão maior o índice de um componente, menor o valor de sua ligação.

Podemos alterar a Equação \ref{Equation:WeightEdge1} para atenuar relações entre
componentes com índices distintos

\begin{equation}
	\label{Equation:WeightEdge2}
	e = \frac{1}{1 + \alpha_3 \cdot d(N_{f(i_a)}, N_{f(i_b)})} \cdot \frac{1}{\alpha_2 \cdot F_{f(i_a)}},
\end{equation}

\noindent
onde $d(N_{f(i_a)}, N_{f(i_b)})$ é a distância euclidiana entre $N_{f(i_a)}$ e 
$N_{f(i_b)}$

\begin{equation}
	\label{Equation:EuclidianDistance}
	d = \sqrt{(N_{f(i_a)} - N_{f(i_b)})^2} = \lvert N_{f(i_a)} - N_{f(i_b)} \rvert.
\end{equation}

\noindent
Tal termo atenuante é inserido com finalidade de valorizar relações entre
componentes semelhantes que pertencem a uma mesma classe hierárquica ou a
classes hierárquicas próximas.

Se substituirmos a Equação \ref{Equation:EuclidianDistance} na Equação
\ref{Equation:WeightEdge2}, obtemos

\begin{equation}
	\label{Equation:WeightEdge3}
	e = \frac{1}{1 + \alpha_3 \cdot \lvert F_{i_a+2} - F_{i_b+2} \rvert} \cdot \frac{1}{\alpha_2 \cdot F_{i_a+2}}.
\end{equation}

Os parâmetros $\alpha_1$, $\alpha_2$ e $\alpha_3$, que aparecem nas equações
\ref{Equation:WeightVertex2} e \ref{Equation:WeightEdge3}, obdecem às seguintes
restrições:

\begin{empheq}[left=\empheqlbrace]{equation}
\begin{split}
	\quad \alpha_1 > 0, \\
	\quad \alpha_2 \geq 1, \\
	\quad \alpha_3 \geq 0.
\end{split}
\end{empheq}

É necessário que $\alpha_1$ seja definido positivo, uma vez que o valor $v$
atribuído a um componente qualquer é definido positivo.
De forma similar, $\alpha_3$ é definido positivo para que se mantenha o termo
atenuante como tal.
O parâmetro $\alpha_2$ precisa maior ou igual a 1 para que valores atribuídos às
arestas estejam no intervalo $[0, 1]$ --- para escalas crescentes cujo menor
valor é menor ou igual a 1.

Ainda se faz necessário definir as operações sobre redes hierárquicas,
heterogêneas e navegáveis em múltiplas escalas.

\subsection{Operações sobre as Redes}
\label{subsec:ModelOperations}

O modelo proposto suporta as operações enunciadas no Quadro \ref{Frame:SupportedOperations}.
Consta nele se determinada operação dispara rotinas de simulação do modelo de
forças elásticas e elétricas e a renderização da rede, bem como se sua aplicação
resulta em uma transformação dos valores de componentes e ligações e, em caso
positivo, se local ou global.

A inserção de componentes externos com índices arbitrários se faz necessária
na hipótese de ser imprática a extração de trilhas de execução de determinado
elemento da execução de um sistema, seja por razões legais ou técnicas. A
remoção de tais componentes se faz necessária para reparar uma inserção
incorreta ou inválida.

O agrupamento de componentes topológicos que formam sequência linear --- sem
bifurcações --- reduz o número de elementos renderizados e o espaço necessário
para rederizar a rede.

O agrupamento de componentes elementares possibilita a redução de informação
sobre análise de acordo com o discernimento do operador. Esta operação deve
conservar todas as ligações e deve ser reversível. O agrupamento de componentes
elementares e topológicos é considerado operação ilegal por que sua soma de
índices é não conservativa.

A supressão de componentes permite identificar os elementos da rede cujas
propriedades são impertinentes à análise do operador. Componentes supressos são
renderizados como pontos.

A edição arbitrária do índice de um componente qualquer altera o valor do
componente, de suas ligações e de ligações para ele.

A edição arbitrária do índice de ligações elementares não altera o valor dos
componentes elementares, apenas de suas ligações.
Esta operação nos permite avaliar visualmente a distorção das redes em função
da força das ligações de componentes elementares.

A troca de escala é uma operação que atua sobre a rede. Ela altera qual série ou
função indexadora serão utilizadas para computar os valores das ligações e
componentes da rede.
Será necessário computar novos valores para todos os componentes e ligações de
uma rede sempre que houver troca de escala ou troca de parâmetros pertinentes à
simulação do modelo de forças elásticas e elétricas.
Armazenar o estado de uma rede para cada escala é uma forma de mitigar este
efeito, se os recursos computacionais necessários estiverem disponíveis.

%As seguintes operações disparam a simulação do modelo de forças elásticas e
%elétricas e a rotina de renderização da rede:
%edição arbitrária do índice de um componente;
%edição arbitrária do índice de ligações elementares;
%troca de escala,

\begin{landscape}
\begin{quadro}[!htb]
    \centering
    \caption{Operações suportadas pelo modelo proposto e entidades sobre as quais atuam tais operações \label{Frame:SupportedOperations}}
    \begin{tabular}{*{1}{|L{14cm}}*{4}{|C{1.5cm}}|}
		\hline
		\multicolumn{1}{|C{14cm}|}{\multirow{7}{*}{\textbf{Operação}}} & \multicolumn{2}{C{3.4cm}|}{\textbf{Dispara}} & \multicolumn{2}{C{3.4cm}|}{\textbf{Transfomação}} \\\cline{2-5}
		 & \rotatebox[origin=c]{70}{\textbf{Simulação}}
		 & \rotatebox[origin=c]{70}{\textbf{Renderização}}
		 & \rotatebox[origin=c]{70}{\textbf{Local}}
		 & \rotatebox[origin=c]{70}{\textbf{Global}} \\
        \hline
        Leitura das redes e suas propriedades 													& & & & \\
        \hline
        Inserção de componentes externos com índices arbitrários e suas ligações				& $\bullet$ & $\bullet$ & $\bullet$ & \\
        \hline
		Remoção de componentes externos com índices arbitrários e suas ligações					& $\bullet$ & $\bullet$ & $\bullet$ & \\
        \hline
        Agrupamento de componentes topológicos em cadeia linear única 							& $\bullet$ & $\bullet$ & $\bullet$ & \\
        \hline
		Agrupamento de componentes elementares													& $\bullet$ & $\bullet$ & $\bullet$ & \\
        \hline
        Suprimir componentes																	& & $\bullet$ & & \\
        \hline
        Edição arbitrária do índice de um componente											& $\bullet$ & $\bullet$ & $\bullet$ & \\
		\hline
		Edição arbitrária do índice de ligações elementares										& $\bullet$ & $\bullet$ & & $\bullet$ \\
        \hline
		Troca de parâmetros																		& $\bullet$ & $\bullet$ & & $\bullet$ \\
        \hline
        Troca de escala																			& $\bullet$ & $\bullet$ & & $\bullet$ \\
        \hline

    \end{tabular}
    \fonte{do autor, 2017}
\end{quadro}
\end{landscape}

\subsection{Simulção do Modelo de Forças Elásticas e Elétricas}
\label{subsec:ModelOperations}

% É PRECISO VERIFICAR SE ESTA PROPOSTA É ORIGINAL
O modelo de forças elásticas e elétricas proposto nesta dissertação é uma
adaptação do modelo apresentado por \citeonline{hu2005efficient}, baseado nos
trabalhos de \citeonline{fruchterman1991graph} e \citeonline{walshaw2000multilevel}.
Acrescentamos termos de forças elásticas de contato e forças dissipativas, bem
como remodelamos as forças elétricas e elástica por ligação.
Sobre os componentes de $\mathcal{N}$ ou $\mathcal{M}$ atuam quatro tipos de
forças, portanto: elétrica, elástica de contato, elástica de ligação e
dissipativa.

Dados dois componentes $i, j$ quaisquer, com os respectivos valores
$v_i, v_j$ e coordenadas centrais $\vec{x}_i, \vec{x}_j$, a força elétrica
repulsiva  $\vec{f}_{elétrica}$ entre tais componentes é determinada pela
equação da Lei de Coulomb

\begin{equation}
	\label{Equation:RepulsiveEletricalForce}
	\vec{f}_{elétrica} = \beta_1 \cdot \frac{v_i \  v_j}{{\lvert \vec{x}_i - \vec{x}_j \rvert}^2}, \qquad \beta_1 \geq 0,
\end{equation}

\noindent
onde $\beta_1$ é constante elétrica.

Seja $r_i, r_j$ os respectivos raios de dois componentes $i, j$. Podemos
detectar a distenção por contato $d$ de tais componentes através da seguinte
análise de casos:

\begin{empheq}[left=\empheqlbrace]{equation}
\begin{split}
	\quad \lvert \vec{x}_i - \vec{x}_j \rvert > r_i + r_j, \qquad d = 0 \\
	\quad \lvert \vec{x}_i - \vec{x}_j \rvert \leq r_i + r_j, \qquad d \geq 0 \\
\end{split}
\end{empheq}

Nos casos em que a distância entre $\vec{x}_i$ e $\vec{x}_j$ é menor
que a soma de seus raios, a distenção por contato $d$ é determinada por

\begin{equation}
	\label{Equation:ContactElasticDistension}
	d = (r_i + r_j) - \lvert \vec{x}_i - \vec{x}_j \rvert,
\end{equation}

\noindent
de modo que a força elástica de contato $\vec{f}_{contato}$ é

\begin{equation}
	\label{Equation:ContactElasticForce}
	\vec{f}_{contato} = \beta_2 \cdot d, \qquad \beta_2 \geq 0,
\end{equation}

\noindent
onde $\beta_2$ representa a constante elástica dos componentes interpenetrados.
Consideramos que todos os componentes possuem uma mesma constante elástica, uma
vez que a inserção desta força no modelo tem como objetivo impedir agregação de
componentes em uma mesma região do espaço.

A força elástica de ligação $\vec{f}_{ligação}$ é determinada pelo seu valor
$e$, constante elástica $\beta_3$ e distância de equilíbrio $x_{eq}$

\begin{equation}
	\label{Equation:LinkElasticForce}
	\vec{f}_{ligação} = \beta_3 \cdot e \cdot (x - x_{eq}), \qquad \beta_3 \geq 0, \quad x_{eq} > 0.
\end{equation}

Note que a força elástica de ligação $\vec{f}_{ligação}$ é repulsiva caso
$x < x_{eq}$, e atrativa caso $x > x_{eq}$.
Consideramos que todas as ligações possuem o mesmo valor para a distância de
equilíbrio $x_{eq}$, dado que seu valor $e$ já considera os componentes
conectados e seus índices.
Uma possibilidade seria determinar o valor da ligações de acordo com a Equação
\ref{Equation:WeightEdge1} e determinar $x_{eq}$ por

\begin{equation}
	\label{Equation:EquilibriumDistanceSuggestion}
	x_{eq} = 1 + k \cdot \lvert F_{i_a+2} - F_{i_b+2} \rvert, \qquad k \geq 0,
\end{equation}

\noindent
mas serão mantidas as Equações \ref{Equation:WeightEdge3} e \ref{Equation:LinkElasticForce}.

A força aplicada $\vec{f}_{aplicada}$ a um componente $i$ em uma rede com $n$
elementos é a soma das forças elétricas repulsivas, de contato e de ligação, nos
casos em que se aplicam:

\begin{equation}
	\label{Equation:AppliedForce}
	\vec{f}_{aplicada} = \sum_{i \neq j}^{n} \vec{f}_{elétrica}
		+ \sum_{i \  | \  j}^{n} \vec{f}_{contato}
		+ \sum_{i \leftrightarrow j}^{n} \vec{f}_{ligação},
\end{equation}

\noindent
onde $i \neq j$ denota pares de componentes distintos, $i \ | \ j$
pares em contato com distenção não nula e positiva, e $i \leftrightarrow j$
pares conectados.
Se exitir múltiplas ligações entre componente $i$ e $j$, $i \rightarrow j$, bem
como entre $j$ e $i$, $i \leftarrow j$, todas contribuirão para o termo que soma
as forças elásticas por ligação que atuam sobre o componente $i$.

A força dissipativa $\vec{f}_{dissipativa}$ é um termo cuja função é atenuar
$\vec{f}_{aplicada}$ e dissipar a energia potencial da rede

\begin{equation}
	\label{Equation:FrictionForce}
	\vec{f}_{dissipativa} = \beta_4 \cdot \vec{f}_{aplicada}, \qquad \beta_4 \geq 0,
\end{equation}

\noindent
de modo que a força resultante $\vec{f}_{r}$ é

\begin{equation}
	\label{Equation:FinalForce1}
	\vec{f}_{r} = \vec{f}_{aplicada} - \vec{f}_{dissipativa},
\end{equation}

\noindent
ou ainda, de forma mais conveniente, por

\begin{equation}
	\label{Equation:FinalForce2}
	\vec{f}_{r} = \vec{f}_{aplicada} \cdot (1 - \beta_4), \qquad \beta_4 \geq 0.
\end{equation}

Uma vez de posse da força resultante sobre um componente qualquer, relacionamos
seu movimento com base na Segunda Lei de Newton

\begin{equation}
	\label{Equation:NewtonSecondLaw}
	\vec{f}_{r} = m \cdot \vec{a}.
\end{equation}

Neste trabalho consideramos $m = 1$ para qualquer componente, pois seu índice já
foi utilizado para calcular seu valor e o valor de suas ligações, de modo que
ele atua direta ou indiretamente sobre os quatro tipos de forças consideradas
neste modelo.

Determinamos, portanto, a posição de um componente pelo mapa iterativo

\begin{equation}
	\label{Equation:IterativePosition}
	\vec{x}_{i+1} = \vec{x}_i + \vec{v}_i \  \Delta t + \frac{1}{2} \  \vec{a} \  {\Delta t}^2,
\end{equation}

\noindent
e sua velocidade através do seguinte mapa iterativo

\begin{equation}
	\label{Equation:IterativeVelocity}
	\vec{v}_{i+1} = \vec{v}_i + \vec{a} \  \Delta t, \qquad \vec{v}_0 = \vec{0}.
\end{equation}

% É PRECISO REFERENCIAR ALGO PARA A ESTIPULAÇÃO DO PARÂMETRO DELTA T
O tamanho do passo de tempo da simulação é definido pelo parâmetro não nulo e
positivo $\Delta t$. 
Baixos valores para $\Delta t$ nos permite obter melhor resultado numérico a um
custo computacional, já que serão necessárias mais iterações para que se atinja
algum critério de parada.
Valores maiores de $\Delta t$ podem gerar erro numérico e instabilidade na rede,
uma vez que forças resultantes exercem sua influência por períodos longos e
podem produzir interpenetração severa. 
Interpenetrações severas devem ser evitadas por que a constante elástica de
contato é alta, de modo que atuam como fonte indevida de energia potencial e
produzem movimentos instáveis na rede.
Iteressante é encontrar valores para $\Delta t$ que minimize o número de
iterações necessárias para atingir um dos critérios de parada e mantenha a
simulação estável.
A precisão numérica, embora atraente, não é importante para a nossa aplicação,
uma vez que posições aproximadas são suficientes para representar a estrutura de
execução das redes estudadas.
Esta dissertação estabelecerá valores para $\Delta t$ empiricamente.

A atualização da posição dos componentes da rede será sícrona, isto é, primeiro
determinaremos as forças sobre todos os componentes, depois atualizaremos suas
coordenadas e seus valores de velocidade.

A simulação obedecerá dois critérios de parada.
O primeiro critério de parada é determinado pelo número máximo de iterações
$\sigma_1 \geq 0$.
O segundo critério de parada é determinado pela norma da maior força resultante
$f_m$ em determinado $\Delta t$.
O processo simulacional é interrompido caso $f_m \leq \sigma_2,\ \sigma_2 \geq 0$.

A posição inicial dos componentes é determinada através de coordenadas em
circunferências concéntricas.
Percorremos a rede em profundidade e classificamos cada elemento de acordo o
número de componentes até a raiz.
Dispomos os elementos de classes iguais em circunferências concéntricas,
igualmente espaçados.
Em ordem crescente, componetes de maior nível são dispostos em circunferências
maiores.
O raio de uma circunferência é estipulados de modo a respeitar distância mínima
$\delta_1$ entre circunferências adjacentes e $\delta_2$ entre componentes.
Este método inviabiliza interpenetração de componentes em $t = 0$.

A área $A$ de um componente é dada pelo inverso de seu valor, multiplicado por
um fator $\gamma_1$

\begin{equation}
	\label{Equation:ComponenteArea}
	A = \gamma_1 \cdot \frac{1}{\alpha_1 \cdot F_{i+2}}, \qquad \gamma_1 > 0,
\end{equation}

\noindent
uma vez que componentes com índices próximos do índice elementar determinam como
se dá a colaboração na rede, ao passo em que os componentes com índices elevados
determinam sua estrutura. 

Seja $A_e$ área de um componente elementar. A área de um componente oculto $A_h$
é:

\begin{equation}
	\label{Equation:HiddenComponenteArea}
	A_h = \gamma_2 \cdot A_e, \qquad  0 < \gamma_2 \leq 1,
\end{equation}

\noindent
com o objetivo de reduzir a área de componentes ocultos. $A_h$ é considerado
limite inferior para a área de um componente.

%Por fim, definiremos aqui uma operação que somente faz sentido no momento em que
%se pode visualizar a rede. Propomos que se possa alterar o valor do índice
%atribuído aos componentes elementares apenas no que tange o cálculo do valor de
%suas ligações. A distorção da rede --- ou a falta dela --- em função das
%ligações dos componentes elementares pode revelvar informações até então não
%percebidas.

%\section{Análises sobre o Modelo}
%\label{subsec:ModelOperations}

%\section{Implementação do Modelo}
%\label{ModelImplementation}

%\subsection{Desafios Técnicos e Tecnológicos}
%\label{TechnicalChallenges}

%\subsection{Requisitos da Solução}
%\label{Requirements}
