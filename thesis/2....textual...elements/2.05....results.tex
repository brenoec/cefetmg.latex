
% Resultados
%

\chapter{Resultados Preliminares}
\label{Chapter:Results}

Neste primeiro estudo de caso preliminar, avaliamos o sistema de software
\texttt{gulp}\footnote{Disponível em: \href{http://gulpjs.com}. Acessado em 22/07/2017.},
cuja estrutura de execução conta com 99 linhas de código, 3 arquivos fontes e 5
funções.
Este sistema foi escolhido por que foi a menor aplicação encontrada no trabalho
de \citeonline{silva2017identifying}, que selecionou os 1000 repositórios mais
populares de aplicações JavaScript de acordo com seu número de estrelas.
Tal seleção ocorreu em 2015.

O \texttt{gulp} é um \textit{task runner} JavaScript que permite automatizar tarefas
rotineiras do fluxo de desenvolvimento, tais quais construir aplicações,
executar testes unitários, entre outras.

Sua estrutura simplificada e $\mathcal{M}_{\texttt{gulp}}$
--- mapa da estrutura de execução do sistema de software \texttt{gulp}, com componentes
topológicos --- se encontram representados nas Figuras
\ref{Figure:GulpStructure} e \ref{Figure:GulpMap}, respectivamente.

\begin{figure}[!htb]
    \centering
    \caption{Estrutura simplificada para a aplicação \texttt{gulp}}
    \includegraphics[scale=1.2]{../shared files/figures/2.05....results/01.1....gulp...structure/main.pdf}
    \fonte{do autor, 2017}
    \label{Figure:GulpStructure}
\end{figure}

\begin{figure}[!htb]
    \centering
    \caption{Mapa $\mathcal{M}_{\texttt{gulp}}$ da estrutura de execução do sistema de software \texttt{gulp}, com componentes topológicos}
    \includegraphics[scale=1.2]{../shared files/figures/2.05....results/01.2....gulp...map/main.pdf}
    \fonte{do autor, 2017}
    \label{Figure:GulpMap}
\end{figure}

A rede foi produzida por análise estática, uma vez que se trata de uma aplicação
hipotética de pequeno porte e o resultado de sua instrumentação é simples de se
obter.
Não utilizamos o processo de simulação de forças elásticas e elétrica proposto
neste trabalho, dado que a implementação da ferramenta de simulação ainda não
foi concluída.
Componentes externos referentes aos módulos diretamente utilizados pela
aplicação avaliada foram adicionados. Dependências indiretas foram omitidas.

O componente anônimo definido por \texttt{index.js} é uma função definida e
no escopo do componente \texttt{watch}.
Isto mostra que a rede --- como definida --- não tem capacidade de representar
este nível de detalhamento.
Evitamos representar exceções e peculiaridades para não pesar o modelo visual
dos mapas renderizados.

%Um problema que é preciso enfrentar sempre que se decide representar
%informações em uma rede é o de como isso afeta a clareza visual da rede.

O \texttt{gulp} provê funcionalidades das mais diversas. No entanto,
$\mathcal{M}_{\texttt{gulp}}$ é uma rede simples, com poucos componentes e 
ligações. Como pode $\mathcal{M}_{\texttt{gulp}}$ prover tantas funcionalidades?
Além de depender de uma série de módulos, \texttt{gulp} é extensível por
intermédio de \textit{plugins}, o que nos faz pensar na necessidade de estudar
não só sistemas de software, mas também a interação entre tais sistemas: análise
de um sistema de sistemas.
%Como pode uma rede tão pequena quanto $\mathcal{M}_{\texttt{gulp}}$ ? Ela não provê. 
