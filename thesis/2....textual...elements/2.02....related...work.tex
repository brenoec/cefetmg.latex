
% Trabalhos Relacionados
%

\chapter{Trabalhos Relacionados}
\label{chap:RelatedWork}

Neste capítulo serão apresentados modelos, ferramentas e metodologia utilizados
para apoiar profissionais nas atividades de desenvolvimento, manutenção e
evolução de software. Serão apresentadas as ferramentas comumente utilizadas ---
como \textit{loggers}, depuradores e \textit{profilers} ---, trabalhos inseridos
nas áreas de Cognição, Compreensão e Visualização de Software, bem como
trabalhos onde a análise dos sistemas de software é feita por meio da Teoria de
Redes Complexas.

\section{\textit{Loggers}, Depuradores e \textit{Profilers}}
\label{sec:Loggers}

\textit{Loggers} são ferramentas que monitoram a execução de sistemas de
software através de registros ou entradas de \textit{log}.
Os registros podem estar dispostos em um formato específico ---
\textit{extensible markup language} (XML), \textit{JavaScript Object Notation} (JSON),
entre outros --- ou constar como texto informativo.
Podem ainda estar associados a níveis distintos --- \textit{information},
\textit{warning}, \textit{error}, entre outros.
As informações podem ser veiculadas a diferentes tipos de mídia, como arquivos
texto, arquivos binários, bases de dados relacionais, não relacionais ou
armazenadas em memória.

A análise de um conjunto de registros usualmente se dá por cálculos estatísticos
e por meio de mecanismos de buscas convencionais na mídia nos quais eles foram
veiculados.
Tais ferramentas possuem diferentes métodos de entradas. Seus recursos podem ser
acessados através de um protocolo específico, de uma
\textit{application program interface} (API) ou de funções e métodos.
São exemplos de \textit{loggers}:
SmartInspect\footnote{Disponível em: \href{http://www.gurock.com/smartinspect/}{http://www.gurock.com/smartinspect/}. Acessado em 18/05/2017.},
Logentries\footnote{Disponível em: \href{http://logentries.com/}{http://logentries.com/}. Acessado em 18/05/2017.} e
Logentries\footnote{Disponível em: \href{http://sentry.io/welcome/}{http://sentry.io/welcome/}. Acessado em 18/05/2017.}.

%\section{Depuradores}
%\label{sec:Debuggers}

Depuradores são ferramentas que possibilitam avaliar a execução de sistemas de
software. Tais ferramentas permitem definir pontos de parada ---
\textit{breakpoints}, \textit{watchpoints} e \textit{catchpoints} ---,
condicionais ou não, nos quais o depurador pausa a execução. Uma vez pausada a
execução do sistema, pode-se executar até o próximo ponto de parada, próxima
expressão, ou adentrar uma definição --- como um método ou função. Permitem ainda
avaliar determinadas expressões e monitorar certos valores de variáveis do
sistema.

Alguns depuradores implementam funcionalidades que permitem a depuração de:
sistemas já em execução; software em execução em máquina remota; programas
\textit{multithread}. Outras funcionalidades e técnicas de depuração são
disponibilizadas por depuradores e desenvolvidas por pesquisadores.
\citeonline{Engblom:ReverseDebuggin:2012} revisou o estado da arte referente à
depuração reversa, que consiste em interromper a execução do software ao se
constatar uma falha e desfazer o histórico de execuções para se avaliar o que a
causou.
São exemplos de depuradores \textit{The GNU Project Debugger} (GDB)
\footnote{Disponível em: \href{http://www.gnu.org/s/gdb/}{http://www.gnu.org/s/gdb/}. Acessado em 18/05/2017.} e 
\textit{The Java Debugger} (JDB)
\footnote{Verifique: \href{http://www.tutorialspoint.com/jdb/}{http://www.tutorialspoint.com/jdb/}. Acessado em 18/05/2017.}.

%\section{\textit{Profilers}}
%\label{sec:Profilers}

\textit{Profilers} são ferramentar utilizadas para extrair dados da execução de
sistemas de software através de técnicas de análise estática e dinâmica.
\citeonline{Thiel:2006:Profiling} analisa 8 ferramentas e discorre sobre
instrumentação em tempo de compilação, ferramentas de amostragem, instrumentação
através de contadores de hardware, instrumentação binária, dentre outras
técnicas.
São exemplos de ferramentas de \textit{profiling} Gprof\footnote{Verifique: \href{https://sourceware.org/binutils/docs/gprof/}{https://sourceware.org/binutils/docs/gprof/}. Acessado em 20/05/2017}
e Dtrace\footnote{Verifique: \href{http://dtrace.org/blogs/about/}{http://dtrace.org/blogs/about/}. Acessado em 20/05/2017}.

\citeonline{Henderson:2017:Software} relatou as principais práticas de
Engenharia de Software praticadas pela empresa Google\footnote{Disponível em:
\href{http://www.google.com/}{http://www.google.com/}. Acessado em 20/05/2012}.
Na seção referente às ferramentas de depuração e \textit{profiling}, Henderson
constata que os servidores da empresa possuem bibliotecas que disponibilizam
determindas ferramentas de análise. No caso de falha, informações da execução do
serviço --- amostradas, em alguns casos --- são salvas em arquivos de
\textit{log}. Menciona ainda sobre uma interface web para depurar 
\textit{remote procedure calls} (RCPs), alterar argumentos de linha de comando,
avaliar uso de recursos computacionais, \textit{profiling}, entre outras coisas.
Discorre brevemente, por fim, da facilidade de se depurar sistemas de software
na empresa e o quão raras são as ocasiões em que se mostra necessário utilizar
um depurador convencional como o GDB.

O modelo proposto neste trabalho irá registrar trilhas de execução em
\textit{log}. A instrumentação do sistema de software estudado pode ser
contínua ou desabilitada --- até mesmo amostrada. A integração do modelo com
a execução do sistema em tempo real não foi considerada, embora tecnicamente
possível. Isso viabilizaria, por exemplo, uma técnica de depuração através da
rede do software.

\section{Cognição, Compreensão e Visualização de Software}
\label{sec:SoftwareCognition}

\citeonline{Pacione:2004:Software} teorizou um modelo com três dimensões de
abstração.
A primeira dimensão trata da granularidade dos elementos de tal
modelo: em um nível de menor granularidade são consideradas as instruções e
expressões; componentes e pacotes representam um nível maior de granularidade.
A segunda dimensão trata das diversas facetas pelos quais os componentes podem
ser avaliados, como por exemplo a análise da estrutura do software, de seu 
comportamento ou do fluxo de dados.
A terceira dimensão trata do tipo de análise empregada, se estática ou dinâmica.
\citeonline{Pacione:2004:NovelModel} criou um modelo com base no modelo
teorizado por \citeonline{Pacione:2004:Software}.

No modelo proposto neste trabalho a granularidade selecionada para análise não
necessariamente é homogênea: enquanto parte do sistema é considerada em nível
maior granularidade, outras partes da rede podem ser expostas de forma
detalhada. Cada componente da rede remete a um ou mais elementos da estrutura
estática do software, enquanto que as ligações revelam as interações entre esses
componentes durante a execução do sistema. O fluxo de dados pode ser
parcialmente observado ao passo em que essas interações ocorrem, uma vez que o
modelo, a priori, não provê mecanismos de monitorar valores.

\citeonline{Hendrix:2002:ControlStructureDiagrams} propos diagramas de
estruturas de controle renderizados juntamente com o código-fonte no editor de
texto e validou a eficácia de tais diagramas através de dois experimentos, com
38 e 50 estudantes.
Os estudantes foram separados em dois grupos --- controle e experimental --- com
o mesmo número de indivíduos, para cada experimento.
A análise estatística rejeitou fortemente a hipótese nula do estudo que
considera não relevante que diagramas de estruturas de controle afetam
positivamente a performance dos entrevistados em responderem perguntas.

O destaque visual de certas informações pode auxiliar na compreensão da rede.

%\section{Visualização de Software}
%\label{sec:SoftwareVisualization}

%\section{Redes Complexas}
%\label{sec:ComplexNetworks}

\section{Análise de Software por Redes Complexas}
\label{sec:SoftwareComplexNetworks}

\citeonline{Valverde:2002:Complex} foram pioneiros ao introduzir a Teoria de
Redes Complexas à análise de sistemas de software, pois abstrairam um diagrama
de classes e modelaram uma rede com ligações não direcionadas entre classes e
métodos. O trabalho analisou 2 sistemas de software e constatou que ambos
apresentam propriedades de redes de mundo pequeno e de redes livre de escala.

\citeonline{Myers:2003:Complex} argumentou que a natureza da relação entre
componentes altera o fluxo de dados e é, portanto, significante, de modo que a
rede modelada deve conter ligações direcionadas entre seus nós. Este trabalho
estudou 3 sistemas orientados a objetos e outros 3 sistemas procedurais através
de grafos de colaboração obtidos por análise estática. Todos os sistemas de
software apresentaram características de redes de mundo pequeno e livres de
escala, despeito de suas diferenças em aspectos tecnológicos e metodológicos.

\citeonline{Valverde:2003:Complex} consideraram uma base de diagramas de classes
de 23 sistemas de software escritos em Java e C++. Modelaram redes com ligações
direcionadas a partir de tal base e observaram que além de tais redes
apresentarem estrutura de redes de mundo pequeno e redes livre de escala,
compartilham modularidade e características hierárquicas.

Certos autores estudaram componentes de sistemas de software em diferentes
níveis de granularidade.

\citeonline{Moura:2003:Signatures} analisou rede formada por cabeçalhos de
sistemas escritos em \texttt{C} e \texttt{C++} e verificou que tais redes
possuem características de rede de mundo pequeno e livre de escala.
Apontam ainda que o crescimento do sistema é o processo que faz com que a rede
seja livre de escala --- processo de ligação preferencial, conforme destacado
por \citeonline{Barabasi:1999:EmergenceScaling} ---, enquanto que o processo de
manutenção perfectiva é responsável pela configuração da rede de mundo pequeno.

\citeonline{LaBelle:2004:PackageNetworks} estudaram a rede de dependências entre
pacotes dos repositórios Debian GNU/Linux e FreeBSD Ports Collection e
identificou características de rede de mundo pequeno e livre de escala.

% EVOLUÇÃO

% METRICAS

% POO & POWERLAW

% TRABALHOS RECENTES
