
% Conclusão
%

\chapter{Conclusão}
\label{Chapter:Conclusion}

Nesta dissertação propomos uma metodologia para modelar e renderizar a estrutura
de execução de sistemas de software através de redes hierárquicas, heterogêneas
e navegáveis em múltiplas escalas. Esta abordagem é agnóstica de linguagens e
paradigmas de programação, além de disponibilizar operações, dentre outras, de
agrupamento e ocultação de componentes.

A mudança de escala índice-série permite analisar componentes com alto nível
hierárquico, que nos informa da estrutura do sistema em análise, ao passo em que
exibe detalhes de sua estrutura de execução.
Outras abordagens que utilizam de trilhas de execução para análise de software
encontram resistência de adoção, entre outros fatores, por: apresentar um grande
montante de informação aos desenvolvedores e arquitetos de software; ou por
remover detalhes ou trilhas inteiras de execução pertinente à análise dos
mesmos.
A rede reduz a quantidade de informação extraída das trilhas de execução através
da aplicação dos operadores de redução espacial $\mathcal{S}$ e temporal
$\mathcal{T}$ e pelas operações de agrupamento e ocultamento de seus elementos.

A valoração dos componentes da rede considera seu nível hierárquico. A valoração
das ligações depende do nível hierárquico do elemento originatário e do elemento
destinatário. Valoriza componentes e ligações de de baixo nível hierárquico, com
o objetivo de expor detalhes da estrutura de execução, bem como desvaloriza
ligações entre componentes de níveis hierárquicos distintos, uma vez que a
ligação entre componentes próximos revelam relações de funcionamento e 
colaboração, no lugar da noção estrutural estabelecida pelas ligações de
componentes distantes. Esta técnica não considera o tipo de ligação entre
componentes da rede, no entanto.

Ainda se faz necessário implementar ferramentas para extrair e tratar as trilhas
de execução, bem como renderizar a rede. Foram considerados casos didáticos para
a aplicação dos operadores de redução espacial $\mathcal{S}$ e temporal
$\mathcal{T}$, bem como avaliado a ferramenta de automatização de tarefas de
desenvolvimento nomeada gulp.

\section{Trabalhos Futuros}
\label{sec:FutureWork}

\textit{Falta escrever.}

%Também deve indicar, se possível e/ou conveniente, como este trabalho pode ser
%estendido ou aprimorado.

\section{Considerações Finais}
\label{sec:FinalConsiderations}

\textit{Parte do que for debatido nos resultados do estudo de caso. Falta
escrever.}
