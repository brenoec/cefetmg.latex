
% Conclusão
%

\chapter{Conclusão}
\label{Chapter:Conclusion}

Nesta dissertação propomos uma metodologia para modelar e renderizar a estrutura
de execução de sistemas de software através de redes hierárquicas, heterogêneas
e navegáveis em múltiplas escalas.
Ela é aplicável a redes homogêneas, no entanto.
Esta abordagem é agnóstica de linguagens de programação.

%A mudança de escala índice-série permite analisar componentes com alto nível
%hierárquico, que nos informa da estrutura do sistema em análise, ao passo em que
%exibe detalhes de sua estrutura de execução.

% referenciar
Segundo \citeonline{ezzati2017multi}, abordagens que utilizam de trilhas de
execução para análise de software geram grandes quantidades de dados, difíceis
de analisar e gerenciar. Extrair informações pertinentes destes dados não é
tarefa trivial.
$\mathcal{M}$ reduz a quantidade de informação extraída das trilhas de execução
através da aplicação dos operadores de redução espacial $\mathcal{S}$ e temporal
$\mathcal{T}$ e pelas operações de agrupamento e supressão de seus elementos.

A valoração dos componentes da rede considera seu nível hierárquico. A valoração
das ligações depende do nível hierárquico do elemento originatário e do elemento
destinatário. Valoriza componentes e ligações de de baixo nível hierárquico, com
o objetivo de expor detalhes da estrutura de execução, bem como desvaloriza
ligações entre componentes de níveis hierárquicos distintos, uma vez que a
ligação entre componentes próximos revelam relações de funcionamento e 
colaboração, no lugar da noção estrutural estabelecida pelas ligações de
componentes distantes. Esta técnica não considera o tipo de ligação entre
componentes da rede, no entanto.

Ainda se faz necessário implementar ferramentas para extrair e tratar as trilhas
de execução, bem como renderizar a rede. Foram considerados casos didáticos para
a aplicação dos operadores de redução espacial $\mathcal{S}$ e temporal
$\mathcal{T}$, bem como avaliado a ferramenta de automatização de tarefas de
desenvolvimento nomeada gulp.

\section{Trabalhos Futuros}
\label{sec:FutureWork}

Como possibilidades de trabalhos futuros, vislumbramos a aplicação da Teoria de
Redes para obter informações relativas às propriedades topológicas e métricas
das redes. São redes de mundo pequeno? Suas estruturas são livres de escala?
Qual é seu coeficiente de agrupamento? Quais são suas métricas de centralidade?
Essas são algumas da muitas perguntas que se pode responder.
Além de extrair relatórios informativos quanto da distribuição de ligações por
componente, relacionar tais métricas com as métricas estabelecidas na Engenharia
de Software.

Existe a possibilidade de extensão do modelo para que ele permita: comparar
diferentes versões de um sistema;
realizar gerência de conhecimento;
rastrear requisitos;
análisar vários aspectos de qualidade;
verificar cobertura de cóodigo;
engenharia de software colaborativa;
auxiliar a gerência de projetos;
e outras muitas possibilidades. 

É necessário que se faça um estudo detalhado referente aos parâmetros do modelo,
tanto na valorização de componetes e ligações, quanto na renderização das redes.
Esse estudo pode abordar ainda o uso de diferente séries numéricas para tal
valoração.
Outro ponto de estudo referente ao modelo é quanto aos métodos de posicionamento
inicial de componentes, uma vez que o método proposto neste trabalho não se
preocupa com minimizar a energia potencial do sistema, ou utilizar o espaço de
forma eficiente, mas apenas para evitar interpenetrações severas nas primeiras
iterações da simulação de forças.

%Algumas das propriedades topológicas das redes: se são redes de mundo pequeno e
%livres de escala, coeficiente de clusterização, e outras.
%Quanto às métricas extraídas das redes, pode-se saber a distribuição de ligações
%por componentes, relação entre ligações de entrada e de saída, coeficiente de
%clusterização, métricas diversas de centralidade, dentre outras mais. Pode-se
%ainda extrair relações entre as métricas estabelecidas na Engenharia de Software
%e na Teoria de Redes.

%Também deve indicar, se possível e/ou conveniente, como este trabalho pode ser
%estendido ou aprimorado.

\section{Considerações Finais}
\label{sec:FinalConsiderations}

O modelo proposto possui limitações. Não apenas por se fazer valer de técnicas
híbridas de análise de sistemas de software, mas porque sua representação não
captura certas características de uma realidade mais complexa. Ora porque as
operações de redução espacial e temporal retiram parte da informação presente
nas trilhas de execuções instrumentadas; ora porque o modelo --- como definido
--- não é capaz de representar certos cenários. Nosso estudo de caso mostrou
que o modelo representa uma função definida no escopo de outra função como se
definida em um arquivo.

Se retomarmos as perguntas de pesquisa realizadas no começo deste texto, podemos
notar que ainda não temos todas as repostas.
É possível afirmar positivamente às perguntas de pesquisa \texttt{\#1} e
\texttt{\#3} (parcialmente), uma vez que modelamos a rede tal como proposto e ela
se mostrou extensível ao passo em que muito do que descrevemos na metodologia
não fazia parte da proposta inicial de trabalho.
A pergunta de pesquisa \texttt{\#2} ainda permanece sem resposta, uma vez que o
único sistema avaliado até o momento é de pequeno porte e a renderização
simulada não foi efetuada.

Para finalizarmos este trabalho ainda se mostra necessário: fundamentar os
conceitos utilizados; implementar ferramenta para extrair trilhas de execução de
sistemas em análise; implementar ferramenta para renderizar redes de tais
sistemas; realizar estudos de caso em número considerável de aplicações; 
discutir resultados encontrados.

%\begin{itemize}
%	\item Fundamentar os conceitos utilizados;
%	\item Implementar ferramenta para extrair trilhas de execução de sistemas em
%	análise;
%	\item Implementar ferramenta para renderizar redes de tais sistemas;
%	\item Realizar estudos de caso em número considerável de aplicações;
%	\item Discutir resultados encontrados.
%\end{itemize}
