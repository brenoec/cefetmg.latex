
% Introdução
%

\chapter{Introdução}
\label{chap:Introdução}

A Engenharia de Software é uma área incipiente do conhecimento, com menos
de um século de história. Consiste em uma disciplina que trata de questões
políticas, culturais, técnicas e tecnológicas dentro de organizações.
Incorporou, primeiramente, interesse por processos, modelos de processos,
planejamento e gestão de projetos; e, posteriormente, pela qualidade de
sistemas de software \cite{Wazlawick2013:Engenharia}.

\citeonline{Dijkstra:1972:HumbleProgrammer} discursou sobre o fenômeno da crise
do software, onde relatou dificuldade de agentes interessados na Engenharia de
Software em cumprir prazos e orçamentos, entregar produtos de boa qualidade e,
por fim, desenvolver, manter e evoluir sistemas de software.

As atividades de desenvolvimento, evolução e manutenção de software têm natureza
cognitiva \cite{Letovsky1987:Cognitive}, uma vez que se faz necessário aplicar
conhecimentos --- técnicos, tecnológicos, entre outros --- para mapear a
concepção de um sistema de software no sistema de fato
\cite{Brooks1983:TheoryComprehension}. Outra forma de se dizer isto é que 
existem camadas intermediárias entre desejos e necessidades e os sistesmas
de software que possivelmente as sanam (Figura \ref{fig:CamadasAbstraçãoSoftware}).

\begin{figure}[!htb]
    \centering
    \caption{Camadas entre desejos e necessidades e os sistemas de software}
    \includegraphics[scale=1.75]{../shared files/figures/software abstraction layers/main.pdf}
    \fonte{do autor, 2017}
    \label{fig:CamadasAbstraçãoSoftware}
\end{figure}

A quinta Lei de Lehman \cite{Lehman1980:Laws}, a Lei da Conservação da
Familiaridade: Complexidade Percebida, determina que a taxa de crescimento de
um sistema é limitada pela quantidade de informação absorvida coletiva e
individualmente \cite{Wazlawick2013:Engenharia}.

De acordo com \citeonline[p.~94]{Crockford2008:JavaScriptGoodParts}:

\begin{citacao}
\textit{Computer programs are the most complex things that humans make. Programs
are made up of a huge number of parts, expressed as functions, statements, and
expressions that are arranged in sequences that must be virtually free of error.
The runtime behavior has little resemblance to the program that implements it.
Software is usually expected to be modified over the course of its productive
life. The process of converting one correct program into a different correct
program is extremely challenging.}
\end{citacao}

Fica caracterizado o problema que é conhecer e alterar as propriedades dos
sistemas de software corretamente; portanto.
Diversas abordagens surgiram ao longo dos anos para tratar tais questões, de
modo que ferramentas, metodologias, processos e modelos foram propostos por
praticantes e estudiosos. Formou-se então os campos de pesquisa nas áreas de
Cognição, Compreensão e Visualização de Software.

O campo de pesquisa referente à Visualização de Software recebeu considerável
atenção de pesquisadores nos últimos 30 anos, onde uma parcela das ferramentas
desenvolvidas foram baseadas em grafos \cite{Storey2006:Theories} e em grafos
hierárquicos \cite{Jahnke2002:Visualizations}.

Grafos hierárquicos atendem ao requisito funcional de abstração, comum entre os
trabalhos da área. Tal requisito considera que pessoas interessadas trabalham
com diferentes níveis de abstração, de forma a reduzir a quantidade de
informação através do agrupamento de elementos \cite{Kienle2007:Requirements}.

\citeonline{Sayama:2015:Complex} argumenta que redes complexas diferem dos
sistemas dinâmicos tradicionais, pois não há a necessidade de componentes se
ligarem uns aos outros de forma homogênea ou regular.
\citeonline{Weifeng:2011:Complex} cita a aplicação da Teoria de Redes Complexas
a sistemas de software para caracterização de sua topologia, modelagem de seu
crescimento, bem como aplicações nas práticas de Engenharia de Software.

%Os trabalhos de \citeonline{Barabasi:19999:EmergenceScaling} e \citeonline{Strogatz:1998:Networks}
%impulsionaram a teoria de redes complexas às mais diversas áreas do
%conhecimento, tal como Biologia, Medicina, Ciências Sociais, Engenharias ---
%Engenharia de Software inclusa ---, entre tantas outras
%\cite{Barabasi:2003:Linked}, \cite{Borner:2007:Network}, \cite{Barabasi:2016:Network}.

Por vezes o estudo da Engenharia de Software se dá através de métodos de análise
estática, dinâmica, ou, como neste trabalho, híbrida.
A estrutura de um software em tempo de execução difere de sua estrutura
estática \cite{Crockford2008:JavaScriptGoodParts}.
Nas palavras de \citeonline[p.~22]{Gamma1995:Design}:

\begin{citacao}
\textit{An object-oriented program's run-time structure often bears little
resemblance to its code structure. The code structure is frozen at compile-time;
it consists of classes in fixed inheritance relationships. A program's run-time
structure consists of rapidly changing networks of communicating objects. In
fact, the two structures are largely independent. Trying to understand one from
the other is like trying to understand the dynamism of living ecosystems from
the static taxonomy of plants and animals, and vice versa.}
\end{citacao}

A análise da estrutura estática de um software permite estudar representações do
código fonte e relações entre seus componentes.
A análise da estrutura dinâmica de um software permite o estudo de seu
comportamento e de seu fluxo de dados, uma vez que certas relações de execução 
somente são conhecidas durante sua execução.
É importante destacar que as duas técnicas são complementares.

Relações entre as áreas do conhecimento da Engenharia de Software (ES), Cognição
e Compreensão de Software (CS) e Visualização de Software (VS), bem como o
emprego de grafos hierárquicos e a Teoria de Redes Complexas (RC) e a análise de
estruturas estáticas (EE) e dinâmicas (ED) de sistesmas de software estão
ilustradas na Figura \ref{fig:RelaçãoConhecimentoTrabalho}.

\begin{figure}[!htb]
    \centering
    \caption{Relações entre as áreas do conhecimento e o trabalho da dissertação}
    \includegraphics[scale=1.75]{../shared files/figures/research area/main.pdf}
    \fonte{do autor, 2017}
    \label{fig:RelaçãoConhecimentoTrabalho}
\end{figure}

\section{Objetivos}
\label{sec:Objetivos}

Esta dissertação tem como objetivo geral modelar a execução de sistemas de
software através de redes hierárquicas, heterogêneas e navegáveis em múltiplas
escalas, com ligações e componentes topológicos. O modelo será construído a
partir da redução espacial e temporal de trilhas instrumentadas durante a
execução do sistema am análise.

\subsection{Perguntas de Pesquisa}
\label{subsec:PerguntasPesquisa}

Conduziremos a dissertação de modo a responder às seguintes perguntas de
pesquisa:

\noindent
\texttt{\textit{PP \#1 -- }}
É possível modelar a execução de sistemas de software de forma hieráquica, com
componentes heterogêneos e com navegabilidade em múltiplas escalas?

\noindent
\texttt{\textit{PP \#2 -- }}
Tal modelo será capaz de expor detalhes pertinentes às atividades de
desenvolvimento, manutenção e evolução de software?

\noindent
\texttt{\textit{PP \#3 -- }}
É possível tornar este modelo independente de linguagem? Ele é extensível?

\subsection{Objetivos Específicos}
\label{subsec:ObjetivosEspecíficos}

Os objetivos específicos desta dissertação serão separados em dois grupos, que
tratarão dos objetivos específicos referentes à proposta do modelo e sua
implementação, haja vista distinção entre o campo teórico e prático.

Quanto aos objetivos específicos referentes à proposta do modelo, determinar
dados de entrada, transformações sobre tais dados para sua construção, operações
e método de renderização.
Quanto aos objetivos específicos referentes à implementação do modelo,
determinar desafios técnicos e tecnológicos, requisitos da solução, aplicações e
limitações.

\section{Organização do trabalho}
\label{sec:OrganizaçãoTrabalho}

No Capítulo \ref{Chapter:RelatedWork} apresentaremos os trabalhos
relacionados: ferramentas, metodologias e modelos utilizados para apoiar
profissionais nas atividades de  desenvolvimento, manutenção e evolução de
software.
Serão apresentadas as ferramentas comumente utilizadas --- como
\textit{loggers}, depuradores e \textit{profilers} ---, trabalhos inseridos nas
áreas de Cognição, Compreensão e Visualização de Software, bem como trabalhos
onde a análise dos sistemas de software é feita por meio da Teoria de Redes
Complexas.

%No Capítulo 3 abordaremos a fundamentação teórica, onde constará o conhecimento
Futuramente constará no texto capítulo no qual abordaremos a fundamentação
teórica, onde apresentaremos o conhecimento base para a compreensão deste
trabalho.
Visitaremos conceitos da Teoria de Grafos e de Redes Complexas.
Em seguida, discutiremos brevemente a história da Engenharia de Software e
os campos de pesquisa da Cognição, Compreensão e Visualização de Software.

No Capítulo \ref{Chapter:Methodology} detalharemos a metodologia deste trabalho.
%, que consta em duas partes: do modelo e da implementação.
Explanaremos quais são os dados de entrada, como o modelo é construído e quais
são as operações definidas sobre as redes.
Em seguida, detalharemos o processo utilizado para renderizar os mapas da
estrutura de execução de sistemas de software.
Futuramente, este capítulo conterá detalhes da implementação de ferramentas para
a coleta das trilhas de execução e suas transfomações, bem como para a
construção da rede, suas operações e renderização.
%Em seguida, abordaremos como cada um desses pontos se manifesta na
%implementação, além de detalhar o processo utilizado para renderizar os mapas da
%estrutura de execução de sistemas.

No Capítulo \ref{Chapter:Results} apresentaremos estudo de caso da aplicação
\texttt{gulp}, um \textit{task runner} JavaScript.
Uma vez finalizadas as implementações e de posse de resultados sólidos, 
discutiremos o modelo proposto quanto às suas aplicações e limitações e
avaliaremos as ameaças à validade do trabalho.
%Avaliaremos ainda as ameaças à validade do trabalho.

No Capítulo \ref{Chapter:Conclusion} apresentaremos as conclusões preliminares
desta dissertação. Enunciaremos possibilidades de trabalhos futuros já
detectadas e as considerações finais.

O Apêndice \ref{Chapter:SpatialTemporalOperators} contém casos didáticos onde
aplicamos nossa metodologia no algoritmo fatorial recursivo e em um sistema
hipotético de loja virtual.
