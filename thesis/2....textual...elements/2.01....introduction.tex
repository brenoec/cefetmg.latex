
% Introdução
%

\chapter{Introdução}
\label{chap:Introdução}

A Engenharia de Software é uma área incipiente do conhecimento, com menos
de um século de história. Consiste em uma disciplina que trata de questões
políticas, culturais, técnicas e tecnológicas dentro de organizações.
Incorporou, primeiramente, interesse por processos, modelos de processos,
planejamento e gestão de projetos; e, posteriormente, pela qualidade de
sistemas de software \cite{Wazlawick2013:Engenharia}.

\citeonline{Dijkstra:1972:HumbleProgrammer} discursou sobre o fenômeno da crise
do software, onde relatou dificuldade de agentes interessados na Engenharia de
Software em cumprir prazos e orçamentos, entregar produtos de boa qualidade e,
por fim, desenvolver, manter e evoluir sistemas de software.

As atividades de desenvolvimento, evolução e manutenção de software têm natureza
cognitiva \cite{Letovsky1987:Cognitive}, uma vez que se faz necessário aplicar
conhecimentos --- técnicos, tecnológicos, entre outros --- para mapear a
concepção de um sistema de software no sistema de fato
\cite{Brooks1983:TheoryComprehension}. Outra forma de se dizer isto é que 
existem camadas intermediárias entre desejos e necessidades e os sistesmas
de software que possivelmente as sanam (Figura \ref{fig:CamadasAbstraçãoSoftware}).

\begin{figure}[!htb]
    \centering
    \caption{Camadas entre desejos e necessidades e os sistemas de software}
    \includegraphics[width=1\textwidth]{../shared files/figures/software abstraction layers/main.pdf}
    \fonte{do autor, 2017}
    \label{fig:CamadasAbstraçãoSoftware}
\end{figure}

A quinta Lei de Lehman \cite{Lehman1980:Laws}, a lei da conservação da
familiaridade: complexidade percebida, determina que a taxa de crescimento de
um sistema é limitada pela quantidade de informação absorvida coletiva e
individualmente \cite{Wazlawick2013:Engenharia}.

De acordo com \citeonline[p.~94]{Crockford2008:JavaScriptGoodParts}:

\begin{citacao}
\textit{Computer programs are the most complex things that humans make. Programs
are made up of a huge number of parts, expressed as functions, statements, and
expressions that are arranged in sequences that must be virtually free of error.
The runtime behavior has little resemblance to the program that implements it.
Software is usually expected to be modified over the course of its productive
life. The process of converting one correct program into a different correct
program is extremely challenging.}
\end{citacao}

Fica caracterizado o problema que é conhecer e alterar as propriedades dos
sistemas de software corretamente; portanto.
Diversas abordagens surgiram ao longo dos anos para tratar tais questões, de
modo que ferramentas, metodologias, processos e modelos foram propostos por
praticantes e estudiosos. Formou-se então, entre outras campos de pesquisa nas
áreas de Cognição, Compreensão e Visualização de Software.

O campo de pesquisa referente à Visualização de Software recebeu considerável
atenção de pesquisadores nos últimos 30 anos, onde uma parcela das ferramentas
desenvolvidas foram baseadas em grafos \cite{Storey2006:Theories} e em grafos
hierárquicos \cite{Jahnke2002:Visualizations}.

Grafos hierárquicos atendem ao requisito funcional de abstração, comum entre os
trabalhos da área. Tal requisito considera que pessoas interessadas trabalham
com diferentes níveis de abstração, de forma a reduzir a quantidade de
informação através do agrupamento de elementos \cite{Kienle2007:Requirements}.

Redes complexas diferem dos sistemas dinâmicos tradicionais, pois não há a
necessidade de componentes se ligarem uns aos outros de forma homogênea e
regular \cite{Sayama:2015:Complex}.
\citeonline{Weifeng:2011:Complex} cita a aplicação da análise de redes complexas
em sistemas de software para sua caracterização, medição e modelagem de
crescimento, bem como aplicações nas práticas de Engenharia de Software.

%Os trabalhos de \citeonline{Barabasi:19999:EmergenceScaling} e \citeonline{Strogatz:1998:Networks}
%impulsionaram a teoria de redes complexas às mais diversas áreas do
%conhecimento, tal como Biologia, Medicina, Ciências Sociais, Engenharias ---
%Engenharia de Software inclusa ---, entre tantas outras
%\cite{Barabasi:2003:Linked}, \cite{Borner:2007:Network}, \cite{Barabasi:2016:Network}.

A estrutura de um software em tempo de execução difere de sua estrutura
estática \cite{Crockford2008:JavaScriptGoodParts}.
Nas palavras de \citeonline[p.~22]{Gamma1995:Design}:

\begin{citacao}
\textit{An object-oriented program's run-time structure often bears little
resemblance to its code structure. The code structure is frozen at compile-time;
it consists of classes in fixed inheritance relationships. A program's run-time
structure consists of rapidly changing networks of communicating objects. In
fact, the two structures are largely independent. Trying to understand one from
the other is like trying to understand the dynamism of living ecosystems from
the static taxonomy of plants and animals, and vice versa.}
\end{citacao}

A análise da estrutura dinâmica de um software permite o estudo de seu
comportamento e de seu fluxo de dados, já que certas propriedades de sistemas de
software somente são conhecidas durante sua execução. A análise da estrutura
estática de um software, por sua vez, permite estudar representações do código
fonte e de certas relações entre seus componentes.

Relações entre as áreas do conhecimento da Engenharia de Software (ES),
Compreensão de Software (CS) e Visualização de Software (VS), bem como o emprego
de grafos hierárquicos e redes complexas (RC) e a análise da estruturas
estáticas (EE) e dinâmicas (ED) de sistesmas de software estão ilustradas na
Figura \ref{fig:RelaçãoConhecimentoTrabalho}.

\begin{figure}[!htb]
    \centering
    \caption{Relações entre as áreas do conhecimento e o trabalho da dissertação}
    \includegraphics[width=1\textwidth]{../shared files/figures/research area/main.pdf}
    \fonte{do autor, 2017}
    \label{fig:RelaçãoConhecimentoTrabalho}
\end{figure}

\section{Objetivos}
\label{sec:Objetivos}

Esta dissertação tem como objetivo geral modelar a execução de sistemas de
software através de redes hierárquicas, heterogêneas e navegáveis em múltiplas
escalas, com ligações e componentes topológicos. O modelo será construído a
partir da redução espacial e temporal de trilhas instrumentadas durante a
execução do sistema am análise.

\subsection{Perguntas de Pesquisa}
\label{subsec:PerguntasPesquisa}

Conduziremos a dissertação de modo a responder às seguintes perguntas de
pesquisa:

\noindent
\texttt{\textit{PP \#1 -- }}
É possível modelar a execução de sistemas de software com componentes
heterogêneos e navegabilidade em múltiplas escalas?

\noindent
\texttt{\textit{PP \#2 -- }}
Tal modelo será capaz de expor detalhes pertinentes às atividades de
desenvolvimento, manutenção e evolução de software?

\noindent
\texttt{\textit{PP \#3 -- }}
É possível tornar este modelo independente de linguagem? Ele é extensível?

\subsection{Objetivos Específicos}
\label{subsec:ObjetivosEspecíficos}

Os objetivos específicos desta dissertação serão separados em dois grupos, que
tratarão dos objetivos específicos referentes à proposta do modelo e sua
implementação, haja vista distinção entre o campo teórico e prático.

Quanto aos objetivos específicos referentes à proposta do modelo, determinar
dados de entrada, transformações sobre tais dados para construção do modelo,
operações sobre o modelo e análises sobre o modelo.

Quanto aos objetivos específicos referentes à implementação do modelo,
determinar desafios técnicos e tecnológicos, requisitos da solução, aplicações e
limitações.

\section{Organização do trabalho}
\label{sec:OrganizaçãoTrabalho}

No Capítulo \hyperref[Chapter:RelatedWork]{2} apresentaremos os trabalhos
relacionados: ferramentas, metodologias e modelos utilizados para apoiar
profissionais nas atividades de  desenvolvimento, manutenção e evolução de
software.
Serão apresentadas as ferramentas comumente utilizadas --- como
\textit{loggers}, depuradores e \textit{profilers} ---, trabalhos inseridos nas
áreas de Cognição, Compreensão e Visualização de Software, bem como trabalhos
onde a análise dos sistemas de software é feita por meio da Teoria de Redes
Complexas.

%No Capítulo 3 abordaremos a fundamentação teórica, onde constará o conhecimento
Futuramente constará aqui capítulo no qual abordaremos a fundamentação teórica,
onde apresentaremos o conhecimento base para a compreensão deste trabalho.
Discutiremos brevemente a história da Engenharia de Software.
Em seguida, visitaremos alguns conceitos da Teoria de Grafos e Redes.
Também trataremos da cognição, compreensão e visualização de software.

No Capítulo \hyperref[Chapter:Methodoly]{3} detalharemos a metodologia deste
trabalho, que consta em duas partes: do modelo e da implementação.
Explanaremos quais são os dados de entrada, como o modelo é construído, como são
feitas as operações sobre as redes e como são realizadas as análises sobre o
modelo. Em seguida, detalharemos o processo utilizado para renderizar os mapas
da estrutura de execução de sistemas de software.
%Em seguida, abordaremos como cada um desses pontos se manifesta na
%implementação, além de detalhar o processo utilizado para renderizar os mapas da
%estrutura de execução de sistemas.

No Capítulo \hyperref[Chapter:Results]{4} apresentaremos um estudo de caso, bem
como analisaremos e discutiremos o modelo proposto quanto às suas aplicações e
limitações.
Avaliaremos ainda as ameaças à validade do trabalho.

No Capítulo \hyperref[Chapter:Conclusion]{5} apresentaremos as conclusões desta
dissertação, enunciaremos possibilidades de trabalhos futuros e as considerações
finais.
