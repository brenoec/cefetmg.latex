
% Template for academic projects at CEFET-MG
%
% Forked from: https://github.com/cfgnunes/latex-cefetmg
%
% Authors: Breno Martins da Costa Corrêa e Souza <breno.ec@gmail.com>
%

\documentclass[
    %twoside,                               % Print two sided
    oneside,                                % Print one sided
]{cefetmg}

\usepackage[
    alf,
    abnt-emphasize=bf,
    bibjustif,
    recuo=0cm,
%    abnt-doi=expand,
%    abnt-url-package=url,
%    abnt-refinfo=yes,
%    abnt-etal-cite=3,
    abnt-etal-list=3,
    abnt-thesis-year=final
]{abntex2cite}                              % References as in ABNT specifications

% used packages
%

\usepackage[utf8]{inputenc}                                 % Codificação do documento
\usepackage[T1]{fontenc}                                    % Seleção de código de fonte
\usepackage{booktabs}                                       % Réguas horizontais em tabelas
\usepackage{color, colortbl}                                % Controle das cores
\usepackage{float}                                          % Necessário para tabelas/figuras em ambiente multi-colunas
\usepackage{graphicx}                                       % Inclusão de gráficos e figuras
\usepackage[space]{grffile}                                 % Permite espaços nos caminhos de arquivos
\usepackage{icomma}                                         % Uso de vírgulas em expressões matemáticas
\usepackage{indentfirst}                                    % Indenta o primeiro parágrafo de cada seção
\usepackage{microtype}                                      % Melhora a justificação do documento
\usepackage{multirow, array}                                % Permite tabelas com múltiplas linhas e colunas
\usepackage{subeqnarray}                                    % Permite subnumeração de equações
\usepackage{verbatim}                                       % Permite apresentar texto tal como escrito no documento, ainda que sejam comandos Latex
\usepackage{amsfonts, amssymb, amsmath}                     % Fontes e símbolos matemáticos
\usepackage{empheq}
\usepackage[mathscr]{eucal}
\usepackage[algoruled, algochapter, portuguese]{algorithm2e}             % Permite escrever algoritmos em português
%\usepackage[scaled]{helvet}                                % Usa a fonte Helvetica
%\usepackage{times}                                         % Usa a fonte Times
%\usepackage{palatino}                                      % Usa a fonte Palatino
\usepackage{lmodern}                                        % Usa a fonte Latin Modern
%\usepackage[bottom]{footmisc}                              % Mantém as notas de rodapé sempre na mesma posição
\usepackage{ae, aecompl}                                    % Fontes de alta qualidade
%\usepackage{latexsym}                                      % Símbolos matemáticos
%\usepackage{lscape}                                        % Permite páginas em modo "paisagem"
%\usepackage{picinpar}                                      % Dispor imagens em parágrafos
%\usepackage{scalefnt}                                      % Permite redimensionar tamanho da fonte
%\usepackage{subfig}                                        % Posicionamento de figuras
%\usepackage{upgreek}                                       % Fonte letras gregas

% Redefine a fonte para uma fonte similar a Arial (fonte Helvetica)
% \renewcommand*\familydefault{\sfdefault}

% PDF file configuration
%

\makeatletter
\hypersetup{
    portuguese,
    colorlinks=true,            % true: "links" coloridos; false: "links" em caixas de texto
    linkcolor=blue,             % Define cor dos "links" internos
    citecolor=blue,             % Define cor dos "links" para as referências bibliográficas
    filecolor=blue,             % Define cor dos "links" para arquivos
    urlcolor=blue,              % Define a cor dos "hiperlinks"
    breaklinks=true,
    pdftitle={\@title},
    pdfauthor={\@author},
    pdfkeywords={abnt, latex, abntex, abntex2}      % To change as needed!
}
\g@addto@macro\normalsize{
  \setlength\abovedisplayskip{0pt}
  \setlength\belowdisplayskip{7pt}
  \setlength\abovedisplayshortskip{0pt}
  \setlength\belowdisplayshortskip{18pt}
}
\makeatother

% Change blue color
\definecolor{blue}{RGB}{41,5,195}

% Change labels
\renewcommand{\algorithmautorefname}{Algoritmo}
\def\equationautorefname~#1\null{Equa\c c\~ao~(#1)\null}

% Create remissive index
\makeindex

\hyphenation{
    qua-dros-cha-ve
    Kat-sa-gge-los
}

% Thesis
%


% Capa
% 
% ATENÇÃO:
% Caso algum campo não se aplique ao seu documento - por exemplo, em seu
% trabalho não houve coorientador - não comente o campo, apenas deixe vazio, 
% assim: \campo{}
%


% Dados do trabalho acadêmico
%

\titulo{Título do Trabalho}
%\title{Title in English}
\subtitulo{Subtítulo do trabalho}
\autor{Nome completo do autor}
\local{Belo Horizonte}
\data{Junho de 2016} % Normalmente se usa apenas mês e ano


% Natureza do trabalho acadêmico
%
% Use apenas uma das opções: Tese (p/ Doutorado), Dissertação (p/ Mestrado) ou
% Projeto de Qualificação (p/ Mestrado ou Doutorado), Trabalho de Conclusão de
% Curso (Graduação)
%

\projeto{Projeto de Qualificação}


% Título acadêmico
%
% Use apenas uma das opções:
% - Se a natureza for Tese, coloque Doutor
% - Se a natureza for Dissertação, coloque Mestre
% - Se a natureza for Projeto de Qualificação, coloque Mestre ou Doutor
% - Se a natureza for Trabalho de Conclusão de Curso, coloque Bacharel
%

\tituloAcademico{Doutor}


% Área de concentração e linha de pesquisa
%
% OBS: indique o nome da área de concentração e da linha de pesquisa do Programa
% de Pós-graduação nas quais este trabalho se insere.
%
% Se a natureza for Trabalho de Conclusão de Curso, deixe ambos os campos vazios
%

\areaconcentracao{Modelagem Matemática e Computacional}
\linhapesquisa{Sistemas Inteligentes}


% Dados da instituição
%
% OBS: a logomarca da instituição deve ser colocada na mesma pasta que foi 
% colocada o documento principal com o nome de "logoInstituicao". O formato pode
% ser: pdf, jpf, eps
%
% Se a natureza for Trabalho de Conclusão de Curso, coloque em "programa' o nome do curso de graduação
%

\instituicao{Centro Federal de Educação Tecnológica de Minas Gerais}
\programa{Programa de Pós-graduação em Modelagem Matemática e Computacional}
%\programa{Curso de Engenharia de Computação}
\logoinstituicao{0.2}{./04-figuras/logo-instituicao.pdf} % \logoinstituicao{<escala>}{<nome do arquivo>}


% Dados do(s) orientador(es)
%

\orientador{Nome do orientador}
%\orientador[Orientadora:]{Nome da orientadora}
\instOrientador{Instituição do orientador}

\coorientador{Nome do coorientador}
%\coorientador[Coorientadora:]{Nome da coorientadora}
\instCoorientador{Instituição do coorientador}

% 
% Folha de Rosto
% 
% Edite este arquivo comentando as linhas que não se aplicam ao tipo de
% documento acadêmico pretendido.
%

% Trabalho de Conclusão de Curso
%\preambulo{{\imprimirprojeto} apresentado ao Curso de Engenharia de Computação do Centro Federal de Educação Tecnológica de Minas Gerais, como requisito parcial para a obtenção do título de {\imprimirtituloAcademico} em Engenharia de Computação.}

% Projeto de qualificação de Mestrado ou Doutorado
\preambulo{{\imprimirprojeto} apresentado ao Programa de \mbox{Pós-graduação} em Modelagem Matemática e Computacional do Centro Federal de Educação Tecnológica de Minas Gerais, como requisito parcial para a obtenção do título de {\imprimirtituloAcademico} em Modelagem Matemática e Computacional.}

% Dissertação de Mestrado
%\preambulo{{\imprimirprojeto} apresentada ao Programa de \mbox{Pós-graduação} em Modelagem Matemática e Computacional do Centro Federal de Educação Tecnológica de Minas Gerais, como requisito parcial para a obtenção do título de {\imprimirtituloAcademico} em Modelagem Matemática e Computacional.}

% Tese de Doutorado
%\preambulo{{\imprimirprojeto} apresentada ao Programa de \mbox{Pós-graduação} em Modelagem Matemática e Computacional do Centro Federal de Educação Tecnológica de Minas Gerais, como requisito parcial para a obtenção do título de {\imprimirtituloAcademico} em Modelagem Matemática e Computacional.}

% 
% Folha de Aprovação
%
% Este documento foi mantido apenas para preservar a paginação do trabalho
% acadêmico final, após a inserção da folha de aprovação fornecida

\textopadraofolhadeaprovacao{Esta folha deverá ser substituída pela cópia digitalizada da folha de aprovação fornecida.}


\begin{document}

\pretextual
\imprimircapa
\imprimirfolhaderosto{}
% \imprimirfolhadeaprovacao{}
% 
% Dedicatória
% 

\begin{dedicatoria}

Dedico este trabalho àqueles que buscam conhecimento e prezam pelo caráter. 

\end{dedicatoria}
              % Dedicatória
% 
% Acknowledgments
%

\begin{agradecimentos}

Agradeço às professoras doutoras Kécia Aline Marques Ferreira e Cristina Duarte
Murta, que além de me recomendarem ao Programa de Pós-Graduação em Modelagem
Matemática e Computacional, me incentivaram e aconselharam a busca pelo
conhecimento e pela pesquisa, tanto nos anos de graduação, quanto nos de
pós-graduação.

Agradeço aos professores doutores Allbens Atman Picardi Faria e Thiago Gomes de
Mattos, pelos ensinamentos no campo da física e da modelagem, que orientaram
grande parte dos esforços deste trabalho.

Agradeço à professora doutora Kécia Aline Marques Ferreira, pela orientação
desde a graduação, e aos porvindouros orientadores do mestrado, professores
doutores Gray Farias Moita e Cristiano Amaral Maffort.

Agradeço aos colegas de curso Fernando Andrade Ducha e Gustavo Henrique Borges
Martins, pelas conversas construtivas, contribuições importantes e pela
disposição. Agradeço aos colegas Rondinelli Leonardo Jorge e Márcio Júnio Guerra
de Aguilar, pelo companheirismo seja dentro de sala ou fora dela.

Agradeço aos meus pais Eros e Junia Mara, pela presença, carinho e suporte sem
os quais dificilmente trilharia este caminho.
Agradeço à minha irmã Iana, pelo ouvido e pela paciência.
Agradeço ao Ramon, pelas sugestões ao longo do trabalho.
Agradeço à minha irmã Maria Flor, por toda sua generosidade e tolerância.
Agradeço ao Madeixa (Henrique), por repetidas vezes mostrar a leveza
da vida e os pequenos prazeres nas pequenas coisas. 
Agradeço ao meu sobrinho Dom, que tão novo me ensina tanto.
Agradeço à minha sogra Lázara, pelas quitandas, quitudes e pelas críticas
construtivas.
Agradeço à minha namorada Fran, que me aconselhou para que eu ingressasse nesta
jornada e transformou minha vida em algo melhor.

Agradeço à Secretaria do Programa de Pós-Graduação em Modelagem Matemática e
Computacional, pelo suporte, tolerância e rápida resolução de problemas
relacionados a equipamentos, documentos e tantas outras questões.

Agradeço à instituição CEFET-MG, que por tantos anos frequentei --- e espero
frequentar --- e contribuiu financeiramente com meus estudos através de bolsa de
mestrado.

\end{agradecimentos}
         % Agradecimentos
% 
% Epígrafe
%

\begin{epigrafe}

\textit{``No more shall the righteous cower before evil.''}
(Iona, Shield of Emeria)

\end{epigrafe}
                % Epígrafe

% Resumo
%
% Escolha de 3 a 6 palavras ou termos que descrevam bem o seu trabalho. As
% palavras-chaves são utilizadas para indexação. A letra inicial de cada palavra
% deve estar em maiúsculas. As palavras-chave são separadas por ponto.
%

\begin{resumo}
    Síntese do trabalho em texto cursivo contendo um único parágrafo. Para uma
    Tese de Doutorado o resumo deve conter, no máximo, 500 palavras. Para uma
    Dissertação de Mestrado o resumo deve conter, no máximo, 250 palavras. Para
    um Projeto de Qualificação o resumo deve conter, no máximo, 200 palavras. O
    resumo é a apresentação clara, concisa e seletiva do trabalho. No resumo 
    deve-se incluir, preferencialmente, nesta ordem:brevíssima introdução ao
    assunto do trabalho de pesquisa (incluindo motivação e justificativa para a
    realização deste trabalho), o que será feito no trabalho (objetivos), como
    ele será desenvolvido (metodologia), quais são os principais resultados
    obtidos ou esperados e a conclusão (compare os resultados com os da
    literatura e destaque as principais contribuições científicas do trabalho.

    \textbf{Palavras-chave}: Modelo Latex. Trabalho acadêmico monográfico.
    Normas ABNT. Outra palavra.
\end{resumo}
            % Resumo na língua vernácula
% 
% Abstract
%

\begin{resumo}[Abstract]
    Software development, maintenance and evolution activities are highly
	complex, since issues pertinent to their practice range from techniques and
	technologies to sociocultural.
	In order to properly change a software system, it is necessary to understand
	the application domain, the technical and technological aspects involved, as
	well as the system itself.
	In this regard, we propose a model for running software analysis through
	hierarchical, heterogeneous and navigable networks at multiple scales, with
	links and topological components.
	As certain properties of software systems are only known during their
	execution, the model is for the most part constructed from execution traces 
	reduced in space and time.
	Static analysis techniques can be used to complement it.
	The result of the application of the model in systems under analysis is a
	map of its execution structure.
	We conducted a case study on application \texttt{gulp}, a JavaScript
	\textit{task runner}.
	Preliminary results indicate the need to analyze software as a system of
	systems.

    \textbf{Keywords}: 
	Complex Software Networks.
	Complex Hierarchical Networks.
	Force-oriented graph drawing.
	Hybrid Software Analysis.
	Software Visualization.
\end{resumo}
            % Resumo em língua inglesa

% Lista de Figuras
%
% Este arquivo não necessita de ser editado. A lista é gerada automaticamente.
%

\pdfbookmark[0]{\listfigurename}{lof}
\listoffigures*
\cleardoublepage
     % Lista de Figuras
% 
% Lista de Tabelas
%
% Este arquivo não necessita de ser editado. A lista é gerada automaticamente.
%

\pdfbookmark[0]{\listtablename}{lot}
\listoftables*
\cleardoublepage
      % Lista de Tabelas
% 
% Lista de Quadros
%
% Este arquivo não necessita de ser editado. A lista é gerada automaticamente.
%

\pdfbookmark[0]{\listofquadrosname}{loq}
\listofquadros*
\cleardoublepage
      % Lista de Quadros

% Lista de Algoritmos
%
% Este arquivo não necessita de ser editado. A lista é gerada automaticamente.
%

\newcommand{\algoritmoname}{Algoritmo}
\renewcommand{\listalgorithmcfname}{Lista de Algoritmos}

\floatname{algocf}{\algoritmoname}
\newlistof{listofalgoritmos}{loa}{\listalgoritmoname}
\newlistentry{algocf}{loa}{0}

\counterwithin{algocf}{chapter}
\renewcommand{\cftalgocfname}{\algoritmoname\space}
\renewcommand*{\cftalgocfaftersnum}{\hfill--\hfill}

\pdfbookmark[0]{\listalgorithmcfname}{loa}
\listofalgorithms
\cleardoublepage
  % Lista de Algoritmos
% 
% Lista de Siglas
%
% Edite a lista acima para definir "todos" os acrônimos e siglas utilizados
% neste trabalho
%

\begin{siglas}
    \item[API]      \textit{Aplication Program Interface}
    \item[JSON]     \textit{JavaScript Object Notation}
    \item[XML]      \textit{Extensible Markup Language}
\end{siglas}
    % Lista de Abreviaturas e Siglas
% 
% Lista de Símbolos
%
% Edite a lista acima para definir "todos" os símbolos utilizados neste
% trabalho.
%

\begin{simbolos}
    \item[$ \Gamma $] Letra grega Gama
    \item[$ \lambda $] Comprimento de onda
    \item[$ \in $] Pertence
\end{simbolos}
     % Lista de Símbolos

% Sumário
%
% Este arquivo não necessita de ser editado. O sumário é gerado automaticamente.
%

\tableofcontents*
\cleardoublepage
                 % Sumário

\textual

% Introdução
%

\chapter{Introdução}
\label{chap:Introdução}

A Engenharia de Software é uma área incipiente do conhecimento, com menos
de um século de história. Consiste em uma disciplina que trata de questões
políticas, culturais, técnicas e tecnológicas dentro de organizações.
Incorporou, primeiramente, interesse por processos, modelos de processos,
planejamento e gestão de projetos; e, posteriormente, pela qualidade de
sistemas de software \cite{Wazlawick2013:Engenharia}.

\citeonline{Dijkstra:1972:HumbleProgrammer} tratou do fenômeno da crise do
software, onde relatou dificuldade de agentes interessados na Engenharia de
Software em cumprir prazos, cumprir orçamentos, entregar produtos de boa
qualidade e, por fim, desenvolver, manter e evoluir sistemas de software.

As atividades de desenvolvimento, evolução e manutenção de software têm natureza
cognitiva \cite{Letovsky1987:Cognitive}, uma vez que se faz necessário aplicar
conhecimentos --- técnicos, tecnológicos, entre outros --- para mapear a
concepção de um sistema de software no sistema de fato
\cite{Brooks1983:TheoryComprehension}. Outra forma de se dizer isto é que 
existem camadas intermediárias entre desejos e necessidades e os sistesmas
de software que possivelmente as sanam (Figura \ref{fig:CamadasAbstraçãoSoftware}).

\begin{figure}[!htb]
    \centering
    \caption{Camadas entre desejos e necessidades e os sistemas de software}
    \includegraphics[width=1\textwidth]{../shared files/figures/software abstraction layers/main.pdf}
    \fonte{do autor, 2017}
    \label{fig:CamadasAbstraçãoSoftware}
\end{figure}

A quinta Lei de Lehman \cite{Lehman1980:Laws}, a lei da conservação da
familiaridade: complexidade percebida, determina que a taxa de crescimento de
um sistema é limitada pela quantidade de informação absorvida coletiva e
individualmente \cite{Wazlawick2013:Engenharia}.

De acordo com \citeonline[p.~94]{Crockford2008:JavaScriptGoodParts}:

\begin{citacao}
\textit{Computer programs are the most complex things that humans make. Programs
are made up of a huge number of parts, expressed as functions, statements, and
expressions that are arranged in sequences that must be virtually free of error.
The runtime behavior has little resemblance to the program that implements it.
Software is usually expected to be modified over the course of its productive
life. The process of converting one correct program into a different correct
program is extremely challenging.}
\end{citacao}

Fica caracterizado o problema que é conhecer e alterar as propriedades dos
sistemas de software corretamente; portanto.

Diversas abordagens surgiram ao longo dos anos para tratar tais questões, de
modo que ferramentas, metodologias, processos e modelos foram propostos por
praticantes e estudiosos. Formou-se então, entre outras campos de pesquisa nas
áreas de Cognição, Compreensão e Visualização de Software.

O campo de pesquisa referente à Visualização de Software recebeu considerável
atenção de pesquisadores nos últimos 30 anos, onde uma parcela das ferramentas
desenvolvidas foram baseadas em grafos \cite{Storey2006:Theories} e em grafos
hierárquicos \cite{Jahnke2002:Visualizations}.

Grafos hierárquicos atendem ao requisito funcional de abstração, comum entre os
trabalhos da área. Tal requisito considera que pessoas interessadas trabalham
com diferentes níveis de abstração, de forma a reduzir a quantidade de
informação através do agrupamento de elementos \cite{Kienle2007:Requirements}.

Redes complexas diferem dos sistemas dinâmicos tradicionais, pois não há a
necessidade de componentes se ligarem uns aos outros de forma homogênea e
regular \cite{Sayama:2015:Complex}.
\citeonline{Weifeng:2011:Complex} cita a aplicação da análise de redes complexas
em sistemas de software para sua caracterização, medição e modelagem de
crescimento, bem como aplicações nas práticas de Engenharia de Software.

%Os trabalhos de \citeonline{Barabasi:19999:EmergenceScaling} e \citeonline{Strogatz:1998:Networks}
%impulsionaram a teoria de redes complexas às mais diversas áreas do
%conhecimento, tal como Biologia, Medicina, Ciências Sociais, Engenharias ---
%Engenharia de Software inclusa ---, entre tantas outras
%\cite{Barabasi:2003:Linked}, \cite{Borner:2007:Network}, \cite{Barabasi:2016:Network}.

A estrutura de um software em tempo de execução difere de sua estrutura
estática \cite{Crockford2008:JavaScriptGoodParts}.
Nas palavras de \citeonline[p.~22]{Gamma1995:Design}:

\begin{citacao}
\textit{An object-oriented program's run-time structure often bears little
resemblance to its code structure. The code structure is frozen at compile-time;
it consists of classes in fixed inheritance relationships. A program's run-time
structure consists of rapidly changing networks of communicating objects. In
fact, the two structures are largely independent. Trying to understand one from
the other is like trying to understand the dynamism of living ecosystems from
the static taxonomy of plants and animals, and vice versa.}
\end{citacao}

A análise da estrutura dinâmica de um software permite o estudo de seu
comportamento e de seu fluxo de dados, já que certas propriedades desses
sistemas de software somente são conhecidas durante sua execução. A análise da
estrutura estática de um software, por sua vez, permite estudar representações
do código fonte e de certas relações entre seus componentes.

Relações entre as áreas do conhecimento da Engenharia de Software (ES),
Compreensão de Software (CS) e Visualização de Software (VS), bem como o emprego
de grafos hierárquicos e redes complexas (RC) e a análise da estruturas
estáticas (EE) e dinâmicas (ED) de sistesmas de software estão ilustradas na
Figura \ref{fig:RelaçãoConhecimentoTrabalho}.

\begin{figure}[!htb]
    \centering
    \caption{Relações entre as áreas do conhecimento e o trabalho da dissertação}
    \includegraphics[width=1\textwidth]{../shared files/figures/research area/main.pdf}
    \fonte{do autor, 2017}
    \label{fig:RelaçãoConhecimentoTrabalho}
\end{figure}

\section{Objetivos}
\label{sec:Objetivos}

Esta dissertação tem como objetivo geral propor um modelo baseado em grafos
hierárquicos para representar a estrutura de execução de sistemas de software
que exponha suas propriedades.
Elementos da rede são componentes de tais sistemas em determinada granularidade,
homogênea ou não.

\subsection{Perguntas de Pesquisa}
\label{subsec:PerguntasPesquisa}

Conduziremos o projeto de pesquisa de modo a responder às seguintes perguntas de
pesquisa:

\noindent
\texttt{\textit{PP \#1 -- }}
Que propriedades de sistemas de software o modelo é capaz de expor?

\noindent
\texttt{\textit{PP \#2 -- }}
Quais são as aplicações e limitações do modelo?

\subsection{Objetivos Específicos}
\label{subsec:ObjetivosEspecíficos}

Os objetivos específicos desta dissertação serão separados em dois grupos, que
tratarão dos objetivos específicos referentes à proposta do modelo e sua
implementação, haja vista distinção entre o campo teórico e prático.

Quanto aos objetivos específicos referentes à proposta do modelo, determinar
dados de entrada, transformações sobre tais dados e construção do modelo,
operações sobre o modelo, análise sobre o modelo, resultados e analise dos
resultados.

Quanto aos objetivos específicos referentes à implementação do modelo,
determinar desafios técnicos e tecnológicos, requisitos da solução, aplicações e
limitações.

\section{Organização do trabalho}
\label{sec:OrganizaçãoTrabalho}

No Capítulo 2 apresentaremos os trabalhos relacionados: ferramentas, metodologia
e modelos utilizados para apoiar profissionais nas atividades de 
desenvolvimento, manutenção e evolução de software. Serão apresentadas as
ferramentas comumente utilizadas --- como \textit{loggers}, depuradores e
\textit{profilers} ---, trabalhos inseridos nas áreas de Cognição, Compreensão
e Visualização de Software, bem como trabalhos onde a análise dos sistemas de
software é feita por meio da Teoria de Redes Complexas.

No Capítulo 3 abordaremos a fundamentação teórica, onde constará o conhecimento
base para a compreensão deste trabalho. Discutiremos brevemente a história da
Engenharia de Software. Em seguida, visitaremos alguns conceitos da Teoria de
Grafos e Redes. Também trataremos da cognição, compreensão e visualização de
software.

No Capítulo 4 discutiremos quais são os requisitos do modelo proposto, bem como
quais critérios utilizar para verificá-lo e validá-lo. Explanaremos o modelo de
estrutura de software em execução através de redes hierárquicas de forma
discursiva e formal.

No Capítulo 5 analisaremos e discutiremos o modelo proposto quanto às suas
aplicações, limitações e demais características. Discutiremos ainda aspectos
sobre possível implementação do método em uma ferramenta de visualização de
software.

No Capítulo 6 apresentaremos as conclusões desta dissertação, enunciaremos
possibilidades de trabalhos futuros e as considerações finais.
                % Introdução

% Trabalhos Relacionados
%

\chapter{Trabalhos Relacionados}
\label{chap:RelatedWork}

Cada capítulo deve conter uma pequena introdução (tipicamente, um ou dois
parágrafos), em seção não numerada, que deve deixar claro o objetivo e o que
será discutido no capítulo, bem como a organização do capítulo.
              % Trabalhos Relacionados
% 
% Fundamentação Teórica
%

\chapter{Fundamentação Teórica}
\label{chap:TheoreticalFoundation}

É uma boa prática iniciar cada novo capítulo com uma breve texto introdutório 
(tipicamente, dois ou três parágrafos) que deve deixar claro o quê será 
discutido no capítulo, bem como a organização do capítulo.
Também servirá ao propósito de "amarrar"{} ou "alinhavar"{} o conteúdo deste 
capítulo com o conteúdo do capítulo imediatamente anterior - neste caso, 
contando com o texto da seção de "Considerações finais"{} do capítulo anterior.
    % Fundamentação Teórica

% Metodologia
%

\chapter{Metodologia}
\label{Chapter:Methodology}

A metodologia deste trabalho consiste na proposição de um modelo da estrutura da
execução de sistemas de software através de redes hierárquicas, heterogêneas e
navegáveis em múltiplas escalas, com ligações e componentes topológicos.
Futuramente, este capítulo conterá detalhes da implementação de ferramentas para
a coleta das trilhas de execução e suas transfomações, bem como para a
construção da rede, suas operações e renderização.

Exemplos da aplicação do modelo a casos didáticos constam no Apêndice
\ref{Chapter:SpatialTemporalOperators}. Conduzimos estudo de caso no qual
avaliamos o sistema \texttt{gulp}, um \textit{task runner} JavaScript, no
Capítulo \ref{Chapter:Results}.

\section{Modelo Proposto}
\label{sec:PropusedModel}

O modelo proposto visa representar a estrutura de execução de sistemas de
software através de uma rede, onde os elementos são componentes desses sistemas
e ligações são relações entre tais componentes.

Os conceitos de componente e ligação aqui definidos são abstratos, uma vez que a
proposta é agnóstica às linguagens de programação.
Se tratarmos de sistemas orientados a objetos, os compontes elementares da rede
serão métodos; se avaliarmos sistemas funcionais, funções.
Outros cenários também são compatíveis com essa abordagem: em JavaScript, por
exemplo, funções são consideradas métodos, caso definidas como propriedades de
um objeto. Neste caso um componente elementar ora é função, ora é método. 

As ligações entre componentes elementares representam relações direcionadas
onde, dado um componente originário $a$ e um componente destinatário $b$,
estabelecemos que $a$ executa $b$.
Utilizaremos o símbolo $\mathcal{N}$ para denotar redes formadas apenas por
componentes elementares e suas ligações.

O modelo prevê a inserção de componentes e ligações topológicas, de modo que as
redes produzidas se tornam heterogêneas e hierárquicas.
Isto ocorre porque as redes passam a conter diferentes tipos de componentes, com
diferentes relações hierárquicas entre si.
São exemplos de componentes topológicos arquivos, pastas, pacotes e outros
elementos estruturais obtidos a partir de análise estática ou dinâmica.
Dado um componente originário de maior nível hierárquico $a$ e um componente
destinatário de menor nível hierárquico $b$, as relações entre $a$ e $b$ podem
ser caracterizadas por palavras como contém, define, usa, entre outros.
Redes que contenham componentes elementares e topológicos e suas ligações serão
denotadas pelo símbolo $\mathcal{M}$.

\subsection{Construção da Rede}
\label{subsec:DataAquisition}

A construção de $\mathcal{N}$ se dá pela redução espacial e temporal de trilhas
de execução. $\mathcal{M}$ é obtido através da inserção de componentes
topológicos em $\mathcal{N}$.
Se considerarmos um componente como uma localidade da estrutura de execução do
sistema, podemos definir o operador de redução espacial --- denotado por
$\mathcal{S}$ --- como uma transformação em uma trilha de execução que unifica
diferentes ocorrências de um mesmo componente.
Se considerarmos um conjunto qualquer de trilhas de execução, o operador de
redução temporal --- denotado por $\mathcal{T}$ --- é a união dessas
trilhas, reduzidas espacialmente ou não, em uma única estrutura.
Observe que $\mathcal{S}$ atua sobre uma trilha de execução, enquanto que
$\mathcal{T}$ atua sobre um conjunto de trilhas.

Seja $X = \{ \ x_1, \ x_2, \ x_3, \ \dots, \ x_n \ \}$ um conjunto de $n$
trilhas de execução, representados por árvores.
Seja $\dot{X} = \{ \ \dot{x_1}, \ \dot{x_2}, \ \dot{x_3}, \ \dots, \ \dot{x_n} \ \}$
como conjunto de $n$ trilhas de execução reduzidas espacialmente, representados
por grafos.
A relação entre $x_i$ e $\dot{x_i}$ se dá por

\begin{equation}
	\dot{x_i} = \mathcal{S}(x_i),
\end{equation}

\noindent
e entre $X$ e $\dot{X}$ por

\begin{equation}
	\dot{X} = \mathcal{S}(X).
\end{equation}

Mapa $\mathcal{N}$ da estrutura de execução de determinado sistema de software
pode ser obtido por

\begin{equation}
	\mathcal{N} = \mathcal{T}(\mathcal{S}(X)) = \bigcup\limits_{i=1}^{n} \mathcal{S}(x_i).
\end{equation}

Os pseudo-códigos referentes aos operadores $\mathcal{S}$ e $\mathcal{T}$
constam nos Algoritmos \hyperref[Algorithm:SpatialReduction]{3.1} e
\hyperref[Algorithm:TemporalReduction]{3.2}, respectivamente. A expressão
\texttt{para cada} no Algoritmo \hyperref[Algorithm:SpatialReduction]{3.2}
corresponde a percorrer todos os nós de uma árvore que representa uma trilha de
execução.

\begin{algorithm}
    \DontPrintSemicolon 

    \label{Algorithm:SpatialReduction}
    \caption{Operador $\mathcal{S}$ de redução espacial sobre trilha de execução}
    \BlankLine

    \Entrada{trilha de execução $\dot{x}$, trilha de execução reduzida espacialmente $x$}
    \Saida{trilha de execução reduzida espacialmente $x$}
    \BlankLine

    \SetKwFunction{FSpatialReduction}{$\mathcal{S}$}
    \SetKwProg{Operator}{Operador}{:}{\Retorna $x$}
    \Operator{\FSpatialReduction{$\dot{x}, x$}}{
        \BlankLine

        \ParaCada {nó $n\;|\; n \in \dot{x}$} {
            seja $f$ nó antecedente de $n$\\

            \BlankLine
            \Se {$n \notin x$} {
                adiciona vértice $n$ ao grafo $x$\\
            }

            \BlankLine
            \Se {aresta $e(f, n) \notin x$ e $f$ não é nulo } {
                adiciona aresta $e(f, n)$ ao grafo $x$\\
            }

            \BlankLine
            \Se {$f$ é nulo } {
                marca $n$ como ponto de entrada\\
            }
        }
        \BlankLine

    }
\end{algorithm}


\begin{algorithm}
    \DontPrintSemicolon

    \label{Algorithm:TemporalReduction}
    \caption{Operador $\mathcal{T}$ de redução temporal sobre trilhas de execução}
    \BlankLine

    \Entrada{conjunto de trilhas de execução $\dot{X}$}
    \Saida{mapa $\mathcal{M}$ da estrutura de execução}
    \BlankLine

    \SetKwFunction{FTemporalReduction}{$\mathcal{T}$}
    \SetKwProg{Operator}{Operador}{:}{\Retorna $\mathcal{M}$}
    \Operator{\FTemporalReduction{$\dot{X}$}}{
        $x \leftarrow$ grafo vazio\\

        \BlankLine
        \ParaCada {trilha de execução $\dot{x}\;|\;\dot{x} \in \dot{X}$} {
            $x \leftarrow \mathcal{S}(\dot{x}, x)$\\
        }
        \BlankLine

        $\mathcal{M} \leftarrow x$
        \BlankLine
    }
\end{algorithm}


Neste ponto basta inserir componentes topológicos em $\mathcal{N}$ para que se
obtenha $\mathcal{M}$.

\subsection{Valores de Componentes e suas Ligações}
\label{subsec:ModelCharacterization}

Atribuímos tipicamente aos componentes elementares o índice $i_e = 1$.
Para componetes topológicos, quão maior o nível de abstração, maior seu índice.
Uma opção é utilizar a sequência ordinal para $n$ classes hierárquicas, onde o
índice de um componente qualquer é o $k$-\textit{ésimo} elemento do conjunto
$\{\ 1,\ 2,\ \dots,\ n\ \}$ de acordo com sua classe.
Determinaremos o índice de um componente topológico como a soma dos índices dos
componentes em que ele se conecta.

Os valores para os componentes e ligações da rede são dependentes de uma função
indexadora $f(i)$ e série numérica $N$, associadas a uma escala $S$

\begin{equation}
	\label{Equation:SetScale}
	S(N, f(i)) = N_{f(i)}.
\end{equation}

A escala linear --- ou natural --- é definda pelo conjunto dos números naturais
maiores que zero

\begin{equation}
	\label{Equation:NaturalSetScale}
	\mathbb{N}_{i}^{*}.
\end{equation}

A escala de Fibonacci é definida como

\begin{equation}
	\label{Equation:FibonacciSetScale}
	F_{i+2}, \qquad F= \{\ 0,\ 1,\ 1,\ 2,\ 3,\ 5,\ 8,\ \dots\ \}.
\end{equation}

Note que a função de índice expressa pela Equação \ref{Equation:FibonacciSetScale}
nos permite desconsiderar os dois primeiros elementos da série de Fibonacci
--- o elemento 0 e o primeiro elemento 1 --- para que não se represente um
componete com valor 0 e componentes em diferentes níveis hierárquicos com o
mesmo valor 1.

O valor $v$ de um componente é

\begin{equation}
	\label{Equation:WeightVertex1}
	v = \alpha_1 \cdot N_{f(i)},
\end{equation}

\noindent
onde \boldsymbol{$\alpha$} é um vetor de parâmetros de ajuste.

Dado um componente originário $a$ e um componente destinatário $b$, com os
respectivos índices $i_a$ e $i_b$, o valor $e$ de uma ligação é determinada por  

\begin{equation}
	\label{Equation:WeightEdge1}
	e = \frac{1}{\alpha_2 \cdot N_{f(i_a)}},
\end{equation}

\noindent
de modo que quão maior o índice de um componente, menor o valor de sua ligação.

Podemos alterar a Equação \ref{Equation:WeightEdge1} para atenuar relações entre
componentes com índices distintos

\begin{equation}
	\label{Equation:WeightEdge2}
	e = \frac{1}{1 + \alpha_3 \cdot d(N_{f(i_a)}, N_{f(i_b)})} \cdot \frac{1}{\alpha_2 \cdot F_{f(i_a)}},
\end{equation}

\noindent
onde $d(N_{f(i_a)}, N_{f(i_b)})$ é a distância euclidiana entre $N_{f(i_a)}$ e 
$N_{f(i_b)}$

\begin{equation}
	\label{Equation:EuclidianDistance}
	d = \sqrt{(N_{f(i_a)} - N_{f(i_b)})^2} = \lvert N_{f(i_a)} - N_{f(i_b)} \rvert.
\end{equation}

\noindent
Tal termo atenuante é inserido com finalidade de valorizar relações entre
componentes semelhantes que pertencem a uma mesma classe hierárquica ou a
classes hierárquicas próximas.

Se substituirmos a Equação \ref{Equation:EuclidianDistance} na Equação
\ref{Equation:WeightEdge2}, obtemos

\begin{equation}
	\label{Equation:WeightEdge3}
	e = \frac{1}{1 + \alpha_3 \cdot \lvert F_{i_a+2} - F_{i_b+2} \rvert} \cdot \frac{1}{\alpha_2 \cdot F_{i_a+2}}.
\end{equation}

Os parâmetros $\alpha_1$, $\alpha_2$ e $\alpha_3$, que aparecem nas equações
\ref{Equation:WeightVertex2} e \ref{Equation:WeightEdge3}, obdecem às seguintes
restrições:

\begin{empheq}[left=\empheqlbrace]{equation}
\begin{split}
	\quad \alpha_1 > 0, \\
	\quad \alpha_2 \geq 1, \\
	\quad \alpha_3 \geq 0.
\end{split}
\end{empheq}

É necessário que $\alpha_1$ seja definido positivo, uma vez que o valor $v$
atribuído a um componente qualquer é definido positivo.
De forma similar, $\alpha_3$ é definido positivo para que se mantenha o termo
atenuante como tal.
O parâmetro $\alpha_2$ precisa maior ou igual a 1 para que valores atribuídos às
arestas estejam no intervalo $[0, 1]$ --- para escalas crescentes cujo menor
valor é menor ou igual a 1.

Ainda se faz necessário definir as operações sobre redes hierárquicas,
heterogêneas e navegáveis em múltiplas escalas.

\subsection{Operações sobre as Redes}
\label{subsec:ModelOperations}

O modelo proposto suporta as operações enunciadas no Quadro \ref{Frame:SupportedOperations}.
Consta nele se determinada operação dispara rotinas de simulação do modelo de
forças elásticas e elétricas e a renderização da rede, bem como se sua aplicação
resulta em uma transformação dos valores de componentes e ligações e, em caso
positivo, se local ou global.

A inserção de componentes externos com índices arbitrários se faz necessária
na hipótese de ser imprática a extração de trilhas de execução de determinado
elemento da execução de um sistema, seja por razões legais ou técnicas. A
remoção de tais componentes se faz necessária para reparar uma inserção
incorreta ou inválida.

O agrupamento de componentes topológicos que formam sequência linear --- sem
bifurcações --- reduz o número de elementos renderizados e o espaço necessário
para rederizar a rede.

O agrupamento de componentes elementares possibilita a redução de informação
sobre análise de acordo com o discernimento do operador. Esta operação deve
conservar todas as ligações e deve ser reversível. O agrupamento de componentes
elementares e topológicos é considerado operação ilegal por que sua soma de
índices é não conservativa.

A supressão de componentes permite identificar os elementos da rede cujas
propriedades são impertinentes à análise do operador. Componentes supressos são
renderizados como pontos.

A edição arbitrária do índice de um componente qualquer altera o valor do
componente, de suas ligações e de ligações para ele.

A edição arbitrária do índice de ligações elementares não altera o valor dos
componentes elementares, apenas de suas ligações.
Esta operação nos permite avaliar visualmente a distorção das redes em função
da força das ligações de componentes elementares.

A troca de escala é uma operação que atua sobre a rede. Ela altera qual série ou
função indexadora serão utilizadas para computar os valores das ligações e
componentes da rede.
Será necessário computar novos valores para todos os componentes e ligações de
uma rede sempre que houver troca de escala ou troca de parâmetros pertinentes à
simulação do modelo de forças elásticas e elétricas.
Armazenar o estado de uma rede para cada escala é uma forma de mitigar este
efeito, se os recursos computacionais necessários estiverem disponíveis.

%As seguintes operações disparam a simulação do modelo de forças elásticas e
%elétricas e a rotina de renderização da rede:
%edição arbitrária do índice de um componente;
%edição arbitrária do índice de ligações elementares;
%troca de escala,

\begin{landscape}
\begin{quadro}[!htb]
    \centering
    \caption{Operações suportadas pelo modelo proposto e entidades sobre as quais atuam tais operações \label{Frame:SupportedOperations}}
    \begin{tabular}{*{1}{|L{14cm}}*{4}{|C{1.5cm}}|}
		\hline
		\multicolumn{1}{|C{14cm}|}{\multirow{7}{*}{\textbf{Operação}}} & \multicolumn{2}{C{3.4cm}|}{\textbf{Dispara}} & \multicolumn{2}{C{3.4cm}|}{\textbf{Transfomação}} \\\cline{2-5}
		 & \rotatebox[origin=c]{70}{\textbf{Simulação}}
		 & \rotatebox[origin=c]{70}{\textbf{Renderização}}
		 & \rotatebox[origin=c]{70}{\textbf{Local}}
		 & \rotatebox[origin=c]{70}{\textbf{Global}} \\
        \hline
        Leitura das redes e suas propriedades 													& & & & \\
        \hline
        Inserção de componentes externos com índices arbitrários e suas ligações				& $\bullet$ & $\bullet$ & $\bullet$ & \\
        \hline
		Remoção de componentes externos com índices arbitrários e suas ligações					& $\bullet$ & $\bullet$ & $\bullet$ & \\
        \hline
        Agrupamento de componentes topológicos em cadeia linear única 							& $\bullet$ & $\bullet$ & $\bullet$ & \\
        \hline
		Agrupamento de componentes elementares													& $\bullet$ & $\bullet$ & $\bullet$ & \\
        \hline
        Suprimir componentes																	& & $\bullet$ & & \\
        \hline
        Edição arbitrária do índice de um componente											& $\bullet$ & $\bullet$ & $\bullet$ & \\
		\hline
		Edição arbitrária do índice de ligações elementares										& $\bullet$ & $\bullet$ & & $\bullet$ \\
        \hline
		Troca de parâmetros																		& $\bullet$ & $\bullet$ & & $\bullet$ \\
        \hline
        Troca de escala																			& $\bullet$ & $\bullet$ & & $\bullet$ \\
        \hline

    \end{tabular}
    \fonte{do autor, 2017}
\end{quadro}
\end{landscape}

\subsection{Simulção do Modelo de Forças Elásticas e Elétricas}
\label{subsec:ModelOperations}

% É PRECISO VERIFICAR SE ESTA PROPOSTA É ORIGINAL
O modelo de forças elásticas e elétricas proposto nesta dissertação é uma
adaptação do modelo apresentado por \citeonline{hu2005efficient}, baseado nos
trabalhos de \citeonline{fruchterman1991graph} e \citeonline{walshaw2000multilevel}.
Acrescentamos termos de forças elásticas de contato e forças dissipativas, bem
como remodelamos as forças elétricas e elástica por ligação.
Sobre os componentes de $\mathcal{N}$ ou $\mathcal{M}$ atuam quatro tipos de
forças, portanto: elétrica, elástica de contato, elástica de ligação e
dissipativa.

Dados dois componentes $i, j$ quaisquer, com os respectivos valores
$v_i, v_j$ e coordenadas centrais $\vec{x}_i, \vec{x}_j$, a força elétrica
repulsiva  $\vec{f}_{elétrica}$ entre tais componentes é determinada pela
equação da Lei de Coulomb

\begin{equation}
	\label{Equation:RepulsiveEletricalForce}
	\vec{f}_{elétrica} = \beta_1 \cdot \frac{v_i \  v_j}{{\lvert \vec{x}_i - \vec{x}_j \rvert}^2}, \qquad \beta_1 \geq 0,
\end{equation}

\noindent
onde $\beta_1$ é constante elétrica.

Seja $r_i, r_j$ os respectivos raios de dois componentes $i, j$. Podemos
detectar a distenção por contato $d$ de tais componentes através da seguinte
análise de casos:

\begin{empheq}[left=\empheqlbrace]{equation}
\begin{split}
	\quad \lvert \vec{x}_i - \vec{x}_j \rvert > r_i + r_j, \qquad d = 0 \\
	\quad \lvert \vec{x}_i - \vec{x}_j \rvert \leq r_i + r_j, \qquad d \geq 0 \\
\end{split}
\end{empheq}

Nos casos em que a distância entre $\vec{x}_i$ e $\vec{x}_j$ é menor
que a soma de seus raios, a distenção por contato $d$ é determinada por

\begin{equation}
	\label{Equation:ContactElasticDistension}
	d = (r_i + r_j) - \lvert \vec{x}_i - \vec{x}_j \rvert,
\end{equation}

\noindent
de modo que a força elástica de contato $\vec{f}_{contato}$ é

\begin{equation}
	\label{Equation:ContactElasticForce}
	\vec{f}_{contato} = \beta_2 \cdot d, \qquad \beta_2 \geq 0,
\end{equation}

\noindent
onde $\beta_2$ representa a constante elástica dos componentes interpenetrados.
Consideramos que todos os componentes possuem uma mesma constante elástica, uma
vez que a inserção desta força no modelo tem como objetivo impedir agregação de
componentes em uma mesma região do espaço.

A força elástica de ligação $\vec{f}_{ligação}$ é determinada pelo seu valor
$e$, constante elástica $\beta_3$ e distância de equilíbrio $x_{eq}$

\begin{equation}
	\label{Equation:LinkElasticForce}
	\vec{f}_{ligação} = \beta_3 \cdot e \cdot (x - x_{eq}), \qquad \beta_3 \geq 0, \quad x_{eq} > 0.
\end{equation}

Note que a força elástica de ligação $\vec{f}_{ligação}$ é repulsiva caso
$x < x_{eq}$, e atrativa caso $x > x_{eq}$.
Consideramos que todas as ligações possuem o mesmo valor para a distância de
equilíbrio $x_{eq}$, dado que seu valor $e$ já considera os componentes
conectados e seus índices.
Uma possibilidade seria determinar o valor da ligações de acordo com a Equação
\ref{Equation:WeightEdge1} e determinar $x_{eq}$ por

\begin{equation}
	\label{Equation:EquilibriumDistanceSuggestion}
	x_{eq} = 1 + k \cdot \lvert F_{i_a+2} - F_{i_b+2} \rvert, \qquad k \geq 0,
\end{equation}

\noindent
mas serão mantidas as Equações \ref{Equation:WeightEdge3} e \ref{Equation:LinkElasticForce}.

A força aplicada $\vec{f}_{aplicada}$ a um componente $i$ em uma rede com $n$
elementos é a soma das forças elétricas repulsivas, de contato e de ligação, nos
casos em que se aplicam:

\begin{equation}
	\label{Equation:AppliedForce}
	\vec{f}_{aplicada} = \sum_{i \neq j}^{n} \vec{f}_{elétrica}
		+ \sum_{i \  | \  j}^{n} \vec{f}_{contato}
		+ \sum_{i \leftrightarrow j}^{n} \vec{f}_{ligação},
\end{equation}

\noindent
onde $i \neq j$ denota pares de componentes distintos, $i \ | \ j$
pares em contato com distenção não nula e positiva, e $i \leftrightarrow j$
pares conectados.
Se exitir múltiplas ligações entre componente $i$ e $j$, $i \rightarrow j$, bem
como entre $j$ e $i$, $i \leftarrow j$, todas contribuirão para o termo que soma
as forças elásticas por ligação que atuam sobre o componente $i$.

A força dissipativa $\vec{f}_{dissipativa}$ é um termo cuja função é atenuar
$\vec{f}_{aplicada}$ e dissipar a energia potencial da rede

\begin{equation}
	\label{Equation:FrictionForce}
	\vec{f}_{dissipativa} = \beta_4 \cdot \vec{f}_{aplicada}, \qquad \beta_4 \geq 0,
\end{equation}

\noindent
de modo que a força resultante $\vec{f}_{r}$ é

\begin{equation}
	\label{Equation:FinalForce1}
	\vec{f}_{r} = \vec{f}_{aplicada} - \vec{f}_{dissipativa},
\end{equation}

\noindent
ou ainda, de forma mais conveniente, por

\begin{equation}
	\label{Equation:FinalForce2}
	\vec{f}_{r} = \vec{f}_{aplicada} \cdot (1 - \beta_4), \qquad \beta_4 \geq 0.
\end{equation}

Uma vez de posse da força resultante sobre um componente qualquer, relacionamos
seu movimento com base na Segunda Lei de Newton

\begin{equation}
	\label{Equation:NewtonSecondLaw}
	\vec{f}_{r} = m \cdot \vec{a}.
\end{equation}

Neste trabalho consideramos $m = 1$ para qualquer componente, pois seu índice já
foi utilizado para calcular seu valor e o valor de suas ligações, de modo que
ele atua direta ou indiretamente sobre os quatro tipos de forças consideradas
neste modelo.

Determinamos, portanto, a posição de um componente pelo mapa iterativo

\begin{equation}
	\label{Equation:IterativePosition}
	\vec{x}_{i+1} = \vec{x}_i + \vec{v}_i \  \Delta t + \frac{1}{2} \  \vec{a} \  {\Delta t}^2,
\end{equation}

\noindent
e sua velocidade através do seguinte mapa iterativo

\begin{equation}
	\label{Equation:IterativeVelocity}
	\vec{v}_{i+1} = \vec{v}_i + \vec{a} \  \Delta t, \qquad \vec{v}_0 = \vec{0}.
\end{equation}

% É PRECISO REFERENCIAR ALGO PARA A ESTIPULAÇÃO DO PARÂMETRO DELTA T
O tamanho do passo de tempo da simulação é definido pelo parâmetro não nulo e
positivo $\Delta t$. 
Baixos valores para $\Delta t$ nos permite obter melhor resultado numérico a um
custo computacional, já que serão necessárias mais iterações para que se atinja
algum critério de parada.
Valores maiores de $\Delta t$ podem gerar erro numérico e instabilidade na rede,
uma vez que forças resultantes exercem sua influência por períodos longos e
podem produzir interpenetração severa. 
Interpenetrações severas devem ser evitadas por que a constante elástica de
contato é alta, de modo que atuam como fonte indevida de energia potencial e
produzem movimentos instáveis na rede.
Iteressante é encontrar valores para $\Delta t$ que minimize o número de
iterações necessárias para atingir um dos critérios de parada e mantenha a
simulação estável.
A precisão numérica, embora atraente, não é importante para a nossa aplicação,
uma vez que posições aproximadas são suficientes para representar a estrutura de
execução das redes estudadas.
Esta dissertação estabelecerá valores para $\Delta t$ empiricamente.

A atualização da posição dos componentes da rede será sícrona, isto é, primeiro
determinaremos as forças sobre todos os componentes, depois atualizaremos suas
coordenadas e seus valores de velocidade.

A simulação obedecerá dois critérios de parada.
O primeiro critério de parada é determinado pelo número máximo de iterações
$\sigma_1 \geq 0$.
O segundo critério de parada é determinado pela norma da maior força resultante
$f_m$ em determinado $\Delta t$.
O processo simulacional é interrompido caso $f_m \leq \sigma_2,\ \sigma_2 \geq 0$.

A posição inicial dos componentes é determinada através de coordenadas em
circunferências concéntricas.
Percorremos a rede em profundidade e classificamos cada elemento de acordo o
número de componentes até a raiz.
Dispomos os elementos de classes iguais em circunferências concéntricas,
igualmente espaçados.
Em ordem crescente, componetes de maior nível são dispostos em circunferências
maiores.
O raio de uma circunferência é estipulados de modo a respeitar distância mínima
$\delta_1$ entre circunferências adjacentes e $\delta_2$ entre componentes.
Este método inviabiliza interpenetração de componentes em $t = 0$.

A área $A$ de um componente é dada pelo inverso de seu valor, multiplicado por
um fator $\gamma_1$

\begin{equation}
	\label{Equation:ComponenteArea}
	A = \gamma_1 \cdot \frac{1}{\alpha_1 \cdot F_{i+2}}, \qquad \gamma_1 > 0,
\end{equation}

\noindent
uma vez que componentes com índices próximos do índice elementar determinam como
se dá a colaboração na rede, ao passo em que os componentes com índices elevados
determinam sua estrutura. 

Seja $A_e$ área de um componente elementar. A área de um componente oculto $A_h$
é:

\begin{equation}
	\label{Equation:HiddenComponenteArea}
	A_h = \gamma_2 \cdot A_e, \qquad  0 < \gamma_2 \leq 1,
\end{equation}

\noindent
com o objetivo de reduzir a área de componentes ocultos. $A_h$ é considerado
limite inferior para a área de um componente.

%Por fim, definiremos aqui uma operação que somente faz sentido no momento em que
%se pode visualizar a rede. Propomos que se possa alterar o valor do índice
%atribuído aos componentes elementares apenas no que tange o cálculo do valor de
%suas ligações. A distorção da rede --- ou a falta dela --- em função das
%ligações dos componentes elementares pode revelvar informações até então não
%percebidas.

%\section{Análises sobre o Modelo}
%\label{subsec:ModelOperations}

%\section{Implementação do Modelo}
%\label{ModelImplementation}

%\subsection{Desafios Técnicos e Tecnológicos}
%\label{TechnicalChallenges}

%\subsection{Requisitos da Solução}
%\label{Requirements}
                 % Metodologia

% Resultados
%

\chapter{Resultados Preliminares}
\label{Chapter:Results}

Como primeiro estudo de caso avaliaremos sistema de software gulp sem considerar
suas dependências, cuja estrutura de execução conta com 99 linhas de código, 3
arquivos fonte e 5 funções. Na sequência, iremos adicionar componentes externos
referentes aos módulos externos utilizados pela aplicação avaliada.

\textit{Falta escrever.}
                     % Resultados

% Conclusão
%

\chapter{Conclusão}
\label{Chapter:Conclusion}

Nesta dissertação propomos uma metodologia para modelar e renderizar a estrutura
de execução de sistemas de software através de redes hierárquicas, heterogêneas
e navegáveis em múltiplas escalas. Esta abordagem é agnóstica de linguagens e
paradigmas de programação, além de disponibilizar operações, dentre outras, de
agrupamento e ocultação de componentes.

A mudança de escala índice-série permite analisar componentes com alto nível
hierárquico, que nos informa da estrutura do sistema em análise, ao passo em que
exibe detalhes de sua estrutura de execução.
Outras abordagens que utilizam de trilhas de execução para análise de software
encontram resistência de adoção, entre outros fatores, por: apresentar um grande
montante de informação aos desenvolvedores e arquitetos de software; ou por
remover detalhes ou trilhas inteiras de execução pertinente à análise dos
mesmos.
A rede reduz a quantidade de informação extraída das trilhas de execução através
da aplicação dos operadores de redução espacial $\mathcal{S}$ e temporal
$\mathcal{T}$ e pelas operações de agrupamento e ocultamento de seus elementos.

A valoração dos componentes da rede considera seu nível hierárquico. A valoração
das ligações depende do nível hierárquico do elemento originatário e do elemento
destinatário. Valoriza componentes e ligações de de baixo nível hierárquico, com
o objetivo de expor detalhes da estrutura de execução, bem como desvaloriza
ligações entre componentes de níveis hierárquicos distintos, uma vez que a
ligação entre componentes próximos revelam relações de funcionamento e 
colaboração, no lugar da noção estrutural estabelecida pelas ligações de
componentes distantes. Esta técnica não considera o tipo de ligação entre
componentes da rede, no entanto.

Ainda se faz necessário implementar ferramentas para extrair e tratar as trilhas
de execução, bem como renderizar a rede. Foram considerados casos didáticos para
a aplicação dos operadores de redução espacial $\mathcal{S}$ e temporal
$\mathcal{T}$, bem como avaliado a ferramenta de automatização de tarefas de
desenvolvimento nomeada gulp.

\section{Trabalhos Futuros}
\label{sec:FutureWork}

\textit{Falta escrever.}

%Também deve indicar, se possível e/ou conveniente, como este trabalho pode ser
%estendido ou aprimorado.

\section{Considerações Finais}
\label{sec:FinalConsiderations}

\textit{Parte do que for debatido nos resultados do estudo de caso. Falta
escrever.}
                  % Conclusão

\postextual

% References
%
% Loads refbase.bib file and extract cited references. This file does not need
% to be edited.
%

\bibliography{./refbase}{}
\bibliographystyle{abntex2-alf} % Define o estilo ABNT para formatar a lista de referências
             % Referências

% Apêndices
%

\begin{apendicesenv}
\partapendices

% Operadores de Redução Espacial $\mathcal{S}$ e Temporal $\mathcal{T}$: 
\chapter{Casos Didáticos}
\label{Chapter:SpatialTemporalOperators}

\textit{Falta escrever.}

\end{apendicesenv}
               % Apêndices
% 
% Anexos
%

\begin{anexosenv}
\partanexos

\chapter{Nome do anexo}     % edite para alterar o título deste anexo
\label{chap:anexoA}

Lembre-se que a diferença entre apêndice e anexo diz respeito à autoria do texto
e/ou material ali colocado.

Caso o material ou texto suplementar ou complementar seja de sua autoria, então
ele deverá ser colocado como um apêndice. Porém, caso a autoria seja de
terceiros, então o material ou texto deverá ser colocado como anexo.

Caso seja conveniente, podem ser criados outros anexos para o seu trabalho
acadêmico. Basta recortar e colar este trecho neste mesmo documento. Lembre-se
de alterar o "label"{} do anexo.

Organize seus anexos de modo a que, em cada um deles, haja um único tipo de
conteúdo. Isso facilita a leitura e compreensão para o leitor do trabalho. É
para ele que você escreve.

\chapter{Dica: nomes no BibTeX}
\label{chap:anexoB}

Se você utiliza LaTeX para a redação de artigos já deve ter se deparado com
algum tipo de problema no modo como o nome dos autores é apresentado no
documento final (pior é quando a "descoberta"{} ocorre depois de já ter
submetido o paper). Muitas vezes é difícil encontrar uma maneira certa de
escrever o nome no arquivo *.bib e garantir que ele seja transcrito corretamente
independente do estilo utilizado. Este texto tem o intuito de discutir o modo
como o BibTeX interpreta o nome dos autores e ajudar na árdua tarefa de
organizar a bibliografia.

Pessoalmente eu prefiro fornecer o nome completo dos meus autores para o BibTeX,
sem abreviações e sem omitir nomes, quando possível. Desse modo, eu dou garantia
que a minha bibliografia irá conter todos os dados para referenciar o autor
independente do estilo utilizado para apresentá-lo. Depois disso, eu
simplesmente espero que o BibTeX faça a abreviação e a colocação dos nomes da
maneira correta de acordo com o estilo indicado. No entanto, para que essa
tarefa seja feita é preciso apresentar os nomes da maneira correta para que a
sua divisão seja feita de forma apropriada.

Para entender como o BibTeX divide um nome, é preciso conhecer antes as diversas
partes que podem compor o nome de uma pessoa, que, a princípio, são: primeiro
nome, nome do meio, ligação, último nome e júnior. A descrição de cada uma
dessas partes é feita a seguir.

\begin{itemize}
    \item \textbf{Primeiro nome:} é o nome da pessoa, geralmente utilizado para
    identificar uma pessoa em um contexto informal. Ex.: Diego, João, Maria etc.
    Em alguns casos o primeiro nome pode ser composto por dois nomes, como Maria
    Ana, Victor Hugo, etc. Nestes casos, deve-se observar como a pessoa utiliza
    o nome para poder diferenciar a segunda parte como Primeiro nome ou Nome do
    meio.

    \item \textbf{Nome do meio:} é o nome que sucede o primeiro nome, mas
    antecede o último nome, geralmente abreviado, por simplicidade. Ex.: Alan
    Mathison Turing, "Mathison"{} é o nome do meio. É comum uma pessoa possuir
    mais do que um nome do meio e também é comum que o nome do meio de alguns
    autores seja desconhecido, devido às abreviações e omissões feitas pelo
    mesmo.

    \item \textbf{Ligação:} também chamado de separador, são as palavras "de"{},
    "da"{}, "do"{}, "e"{}, "von"{}, entre outras que ligam um nome ao outro. Em
    John von Neumann e Ricardo Luis de Azevedo da Rocha, por exemplo, as
    palavras "von"{}, "de"{}  e "da"{}  são as ligações. Num contexto geral,
    elas normalmente são grafadas com inicial minúscula para não serem
    confundidas com o nome do meio e, embora não seja comum em todos lugares do
    mundo, no Brasil é comum um nome possuir até mais do que uma ligação.

    \item \textbf{Último nome:} também chamado de nome de família, é o nome
    utilizado para identificar uma pessoa em situações formais, como referência
    em artigos, livros etc. Ex.: Albert Einstein, "Einstein"{} é o último nome.

\item \textbf{Júnior:} é um sufixo do nome que indica a existência de um parente
com o mesmo nome. Geralmente abreviado como "Jr."{} pode ser apresentado de
diversas formas como "Filho"{}, "Neto"{} ou traduzido para o idioma de origem
do dono do nome, como "fils"{} (filho) em francês. Ex.: John Forbes Nash Jr.
\end{itemize}

Quando indicamos o nome de um autor no BibTeX ele interpreta os nomes seguindo
uma das três regras a seguir:

\begin{enumerate}
    \item \textbf{Nenhuma vírgula:} {Primeiro nome} {ligação} {Último nome}

    \item \textbf{Uma vírgula:} {ligação} {Último nome}, {Primeiro nome}

    \item \textbf{Duas vírgulas:} {ligação} {Último nome}, {Júnior}, {Primeiro nome}
\end{enumerate}

Como pode-se notar, a distinção entre essas três possíveis interpretações se dá
com base na quantidade de vírgulas que foram inseridas e no posicionamento da
ligação, que devem sempre ser escritas com a inicial minúscula. O(s) nome(s) do
meio são todos os nomes que estão após o primeiro nome, porém antes da ligação e
do último nome. A princípio, o BibTeX interpreta os nomes do meio como sendo
parte do primeiro nome.

Para mostrar como isso pode gerar problemas, imagine, por exemplo, se o nome
"John Forbes Nash Jr."{} fosse apresentado em um arquivo BibTeX. Como nenhuma
vírgula foi inserida, será entendido que "John Forbes Nash"{} é o primeiro nome
e "Jr."{} é o último nome, o que não seria correto. De forma semelhante, se for
apresentado na forma "Nash Jr., John Forbes", então "John Forbes"{} será o
primeiro nome enquanto "Nash Jr."{} será o último nome, que também está
incorreto.

Portanto, a maneira correta de referenciar seria utilizando a terceira opção
pois é a única que inclui o Jr. (utilizando duas vírgulas):
"Nash, Jr., John Forbes"{}, fazendo com que "John Forbes"{} seja compreendido
como primeiro nome, "Nash"{} como último nome e "Jr."{} como o júnior.

Outro grande problema ocorre quando um nome possui mais do que uma ligação, como
em "Ricardo Luis de Azevedo da Rocha"{}. Quando o BibTeX lê um nome como esse,
ele entende que tudo que vem após o ligador, faz parte do último nome. Neste
caso, "Ricardo Luis"{} seria tratado como o primeiro nome e "Azevedo da Rocha"{}
como último nome.

Para evitar esse comportamento, devemos optar pela segunda opção (utilizando uma
vírgula), ou seja, "da Rocha, Ricardo {Luis de} Azevedo"{}, fazendo com que o
último nome seja somente "Rocha"{} e precedido pelo seu ligador.

Note que neste último exemplo o ligador e o nome que o antecede foram
delimitados por chaves. Este é um pequeno e útil truque que pode ser feito para
garantir que os ligadores não sejam inclusos ao abreviar nomes (Ex.:
Universidade de São Paulo, abrevia-se U.S.P. ao invés de U. de S.P. ou
U.d.S.P.). Fazendo isso, o BibTeX passa a tratar "Luis de"{} como um único nome
e o abrevia corretamente quando necessário.

E qual a importância de garantir que o BibTeX interprete corretamente as
diversas partes de um nome? A verdade é que cada estilo trata o nome de uma
maneira diferente: o IEEE, por exemplo, coloca apenas as iniciais do primeiro
nome e a ligação seguida do último nome; a Nature, por outro lado, coloca a
ligação e o último nome, seguido das iniciais do primeiro nome; e assim por
diante. Assim sendo, entender como os nomes são interpretados nos ajuda a
garantir que o mesmo seja sempre dividido da maneira correta e formatado
apropriadamente independente do estilo fornecido.

Por fim, e não menos importante, também deixo aqui um aviso sobre a acentuação
no BibTeX. Eu já presenciei diversos problemas com relação a acentuação nos
nomes dos autores, títulos dos artigos etc. Em especial os problemas ocorreram
quando eu estava utilizando o abnTeX, que é um projeto que tem o objetivo de
implementar o padrão ABNT em formato TeX. Embora este projeto não seja um dos
mais ativos, ele ainda é muito utilizado e alguns grupos de pesquisa utilizam
estilos que nada mais são do que versões derivadas deste (como é o caso do
laboratório que faço parte).

O problema é que este estilo possui uma falha (descrita em
\href{http://abntex.codigolivre.org.br/node5.html}{http://abntex.codigolivre.org.br}),
que impede que acentos sejam convertidos corretamente em letras maiúsculas. Para
contornar o problema eles pedem que sejam utilizados códigos para descrever os
acentos nos arquivos *.bib ao invés de inseri-los diretamente pelo teclado. Dado
a quantidade de problemas que essa falha me gerou, julgo isso como uma boa
prática e deixo aqui a minha recomendação de que não sejam utilizados caracteres
não-ASCII nos arquivos *.bib.

Como os arquivos *.bib são interpretados pelo LaTeX, é possível utilizar alguns
comandos em seus campos. A saber, segue os comandos para formar os acentos mais
comuns:
\\
\\
\\

[A parte final do texto original foi suprimida, por conter incorreções.]
\footnote{Nesta parte era apresentado os comandos \LaTeX{} para acentuação. No
entanto, foi constatado que os comandos, se utilizados como apresentado,
provocariam erros na transformação de minúsculas para maiúsculas e vice-versa,
algo bastante recorrente no estilo \texttt{abntex2}. Para a tabela com os
comandos corretos veja \autoref{fig:acentos-latex}.}.
\\
\\
\\

[Em \href{http://en.wikibooks.org/wiki/LaTeX/Special_Characters}
{http://en.wikibooks.org/wiki/LaTeX/Special\underline{ } Characters}] você
encontra diversos outros acentos e símbolos para serem utilizados no LaTeX.

Referência:\\

Alexander Binder. Help On BibTeX Names. Disponível em
<\href{www.kfunigraz.ac.at/~binder/texhelp/bibtx-23.html}{www.kfunigraz.ac.at/...}>.
Acessado em 4 de março de 2011.

\end{anexosenv}
                  % Anexos
% 
% Índice Remissivo
%
%
% Este comando gera automaticamente o índice remissivo para os termos definidos
% no corpo do documento.
%
% Este arquivo não necessita de ser editado.
%

\printindex
      % Índice Remissivo

\end{document}
