
% Template for academic projects at CEFET-MG
%
% Forked from: https://github.com/cfgnunes/latex-cefetmg
%
% Authors: Breno Martins da Costa Corrêa e Souza <breno.ec@gmail.com>
%

\documentclass[
    %twoside,                               % Print two sided
    oneside,                                % Print one sided
]{cefetmg}

\usepackage[
    alf,
    abnt-emphasize=bf,
    bibjustif,
    recuo=0cm,
%    abnt-doi=expand,
%    abnt-url-package=url,
%    abnt-refinfo=yes,
%    abnt-etal-cite=3,
    abnt-etal-list=3,
    abnt-thesis-year=final
]{abntex2cite}                              % References as in ABNT specifications

% used packages
%

\usepackage[utf8]{inputenc}                                 % Codificação do documento
\usepackage[T1]{fontenc}                                    % Seleção de código de fonte
\usepackage{booktabs}                                       % Réguas horizontais em tabelas
\usepackage{color, colortbl}                                % Controle das cores
\usepackage{float}                                          % Necessário para tabelas/figuras em ambiente multi-colunas
\usepackage{graphicx}                                       % Inclusão de gráficos e figuras
\usepackage[space]{grffile}                                 % Permite espaços nos caminhos de arquivos
\usepackage{icomma}                                         % Uso de vírgulas em expressões matemáticas
\usepackage{indentfirst}                                    % Indenta o primeiro parágrafo de cada seção
\usepackage{microtype}                                      % Melhora a justificação do documento
\usepackage{multirow, array}                                % Permite tabelas com múltiplas linhas e colunas
\usepackage{subeqnarray}                                    % Permite subnumeração de equações
\usepackage{verbatim}                                       % Permite apresentar texto tal como escrito no documento, ainda que sejam comandos Latex
\usepackage{amsfonts, amssymb, amsmath}                     % Fontes e símbolos matemáticos
\usepackage{empheq}
\usepackage[mathscr]{eucal}
\usepackage[algoruled, algochapter, portuguese]{algorithm2e}             % Permite escrever algoritmos em português
%\usepackage[scaled]{helvet}                                % Usa a fonte Helvetica
%\usepackage{times}                                         % Usa a fonte Times
%\usepackage{palatino}                                      % Usa a fonte Palatino
\usepackage{lmodern}                                        % Usa a fonte Latin Modern
%\usepackage[bottom]{footmisc}                              % Mantém as notas de rodapé sempre na mesma posição
\usepackage{ae, aecompl}                                    % Fontes de alta qualidade
%\usepackage{latexsym}                                      % Símbolos matemáticos
%\usepackage{lscape}                                        % Permite páginas em modo "paisagem"
%\usepackage{picinpar}                                      % Dispor imagens em parágrafos
%\usepackage{scalefnt}                                      % Permite redimensionar tamanho da fonte
%\usepackage{subfig}                                        % Posicionamento de figuras
%\usepackage{upgreek}                                       % Fonte letras gregas

% Redefine a fonte para uma fonte similar a Arial (fonte Helvetica)
% \renewcommand*\familydefault{\sfdefault}

% PDF file configuration
%

\makeatletter
\hypersetup{
    portuguese,
    colorlinks=true,            % true: "links" coloridos; false: "links" em caixas de texto
    linkcolor=blue,             % Define cor dos "links" internos
    citecolor=blue,             % Define cor dos "links" para as referências bibliográficas
    filecolor=blue,             % Define cor dos "links" para arquivos
    urlcolor=blue,              % Define a cor dos "hiperlinks"
    breaklinks=true,
    pdftitle={\@title},
    pdfauthor={\@author},
    pdfkeywords={abnt, latex, abntex, abntex2}      % To change as needed!
}
\g@addto@macro\normalsize{
  \setlength\abovedisplayskip{0pt}
  \setlength\belowdisplayskip{7pt}
  \setlength\abovedisplayshortskip{0pt}
  \setlength\belowdisplayshortskip{18pt}
}
\makeatother

% Change blue color
\definecolor{blue}{RGB}{41,5,195}

% Change labels
\renewcommand{\algorithmautorefname}{Algoritmo}
\def\equationautorefname~#1\null{Equa\c c\~ao~(#1)\null}

% Create remissive index
\makeindex

\hyphenation{
    qua-dros-cha-ve
    Kat-sa-gge-los
}

% Thesis
%

% -----------------------------------------------------------------------------
% Capa
% -----------------------------------------------------------------------------

% -----------------------------------------------------------------------------
% ATENÇÃO:
% Caso algum campo não se aplique ao seu documento - por exemplo, em seu trabalho
% não houve coorientador - não comente o campo, apenas deixe vazio, assim: \campo{}
% -----------------------------------------------------------------------------

% -----------------------------------------------------------------------------
% Dados do trabalho acadêmico
% -----------------------------------------------------------------------------

%\title{Title in English}
\titulo{Análise de Software em Execução Através de Redes Hierárquicas, Heterogêneas e Navegáveis em Múltiplas Escalas}
\subtitulo{}
\autor{Breno Martins da Costa Corrêa e Souza}
\local{Belo Horizonte}
\data{Março de 2017} % Normalmente se usa apenas mês e ano

% -----------------------------------------------------------------------------
% Natureza do trabalho acadêmico
% -----------------------------------------------------------------------------

\projeto{Projeto de Qualificação}

% -----------------------------------------------------------------------------
% Título acadêmico
% Use apenas uma das opções:
% - Se a natureza for Tese, coloque Doutor
% - Se a natureza for Dissertação, coloque Mestre
% - Se a natureza for Projeto de Qualificação, coloque Mestre ou Doutor conforme o caso
% - Se a natureza for Trabalho de Conclusão de Curso, coloque Bacharel
% -----------------------------------------------------------------------------

\tituloAcademico{Mestre}

% -----------------------------------------------------------------------------
% Área de concentração e linha de pesquisa
% -----------------------------------------------------------------------------

\areaconcentracao{Modelagem Matemática e Computacional}
\linhapesquisa{Engenharia de Software}

% -----------------------------------------------------------------------------
% Dados da instituição
% -----------------------------------------------------------------------------

\instituicao{Centro Federal de Educação Tecnológica de Minas Gerais}
\programa{Programa de Pós-graduação em Modelagem Matemática e Computacional}
\logoinstituicao{0.2}{./4....figures/logo-instituicao.pdf} % \logoinstituicao{<escala>}{<nome do arquivo>}

% -----------------------------------------------------------------------------
% Dados do(s) orientador(es)
% -----------------------------------------------------------------------------

\orientador{Gray Farias Moita}
%\orientador[Orientadora:]{Nome da orientadora}
\instOrientador{CEFET-MG}

\coorientador{Cristiano Amaral Maffort}
%\coorientador[Coorientadora:]{Nome da coorientadora}
\instCoorientador{CEFET-MG}
% \include{./1....pre-textual...elements/1.01....cover...sheet}
% 
% Folha de Aprovação
%
% Este documento foi mantido apenas para preservar a paginação do trabalho
% acadêmico final, após a inserção da folha de aprovação fornecida

\textopadraofolhadeaprovacao{Esta folha deverá ser substituída pela cópia digitalizada da folha de aprovação fornecida.}


\begin{document}

\pretextual
\imprimircapa
\imprimirfolhaderosto{}
% \imprimirfolhadeaprovacao{}
% 
% Dedicatória
% 

\begin{dedicatoria}

Dedico este trabalho àqueles que buscam conhecimento e prezam pelo caráter. 

\end{dedicatoria}
              % Dedicatória
% 
% Acknowledgments
%

\begin{agradecimentos}

Agradeço às professoras doutoras Kécia Aline Marques Ferreira e Cristina Duarte
Murta, que além de me recomendarem ao Programa de Pós-Graduação em Modelagem
Matemática e Computacional, me incentivaram e aconselharam a busca pelo
conhecimento e pela pesquisa, tanto nos anos de graduação, quanto nos de
pós-graduação.

Agradeço aos professores doutores Allbens Atman Picardi Faria e Thiago Gomes de
Mattos, pelos ensinamentos no campo da física e da modelagem, que orientaram
grande parte dos esforços deste trabalho.

Agradeço à professora doutora Kécia Aline Marques Ferreira, pela orientação
desde a graduação, e aos porvindouros orientadores do mestrado, professores
doutores Gray Farias Moita e Cristiano Amaral Maffort.

Agradeço aos colegas de curso Fernando Andrade Ducha e Gustavo Henrique Borges
Martins, pelas conversas construtivas, contribuições importantes e pela
disposição. Agradeço aos colegas Rondinelli Leonardo Jorge e Márcio Júnio Guerra
de Aguilar, pelo companheirismo seja dentro de sala ou fora dela.

Agradeço aos meus pais Eros e Junia Mara, pela presença, carinho e suporte sem
os quais dificilmente trilharia este caminho.
Agradeço à minha irmã Iana, pelo ouvido e pela paciência.
Agradeço ao Ramon, pelas sugestões ao longo do trabalho.
Agradeço à minha irmã Maria Flor, por toda sua generosidade e tolerância.
Agradeço ao Madeixa (Henrique), por repetidas vezes mostrar a leveza
da vida e os pequenos prazeres nas pequenas coisas. 
Agradeço ao meu sobrinho Dom, que tão novo me ensina tanto.
Agradeço à minha sogra Lázara, pelas quitandas, quitudes e pelas críticas
construtivas.
Agradeço à minha namorada Fran, que me aconselhou para que eu ingressasse nesta
jornada e transformou minha vida em algo melhor.

Agradeço à Secretaria do Programa de Pós-Graduação em Modelagem Matemática e
Computacional, pelo suporte, tolerância e rápida resolução de problemas
relacionados a equipamentos, documentos e tantas outras questões.

Agradeço à instituição CEFET-MG, que por tantos anos frequentei --- e espero
frequentar --- e contribuiu financeiramente com meus estudos através de bolsa de
mestrado.

\end{agradecimentos}
         % Agradecimentos
% 
% Epígrafe
%

\begin{epigrafe}

\textit{``No more shall the righteous cower before evil.''}
(Iona, Shield of Emeria)

\end{epigrafe}
                % Epígrafe

% Resumo
%
% Escolha de 3 a 6 palavras ou termos que descrevam bem o seu trabalho. As
% palavras-chaves são utilizadas para indexação. A letra inicial de cada palavra
% deve estar em maiúsculas. As palavras-chave são separadas por ponto.
%

\begin{resumo}
    Atividades de desenvolvimento, manutenção e evolução de software apresentam
    elevada complexidade, uma vez que questões pertinentes à sua prática vão de
    técnicas e tecnologias a questões socioculturais. Certas propriedades de
    sistemas de software somente são conhecidas durante sua execução. Poucos são
    os trabalhos que tratam da análise de sua estrutura dinâmica; no entanto. O
    presente trabalho propõe um modelo que aplica união e redução em trilhas de
    execução e gera uma rede de componentes interconectados e colaborativos. É
    esperado que tais redes representem com fidelidade a estrutura do software
    em execução. Serão discutidas aplicações e limitações, possível
    implementação, dificuldades técnicas e tecnológicas, bem como possibilidades
    colaborativas na Engenharia de Software de tal modelo.

    \textbf{Palavras-chave}: Redes Complexas Hierárquicas.
    Análise Dinâmica de Software. Visualização de Software. Modelos de Software.
    Engenharia de Software Colaborativa.
\end{resumo}
            % Resumo na língua vernácula
% 
% Abstract
%

\begin{resumo}[Abstract]
    Software development, maintenance and evolution activities are highly
	complex, since issues pertinent to their practice range from techniques and
	technologies to sociocultural.
	In order to properly change a software system, it is necessary to understand
	the application domain, the technical and technological aspects involved, as
	well as the system itself.
	In this regard, we propose a model for running software analysis through
	hierarchical, heterogeneous and navigable networks at multiple scales, with
	links and topological components.
	As certain properties of software systems are only known during their
	execution, the model is for the most part constructed from execution traces 
	reduced in space and time.
	Static analysis techniques can be used to complement it.
	The result of the application of the model in systems under analysis is a
	map of its execution structure.
	We conducted a case study on application \texttt{gulp}, a JavaScript
	\textit{task runner}.
	Preliminary results indicate the need to analyze software as a system of
	systems.

    \textbf{Keywords}: 
	Complex Software Networks.
	Complex Hierarchical Networks.
	Force-oriented graph drawing.
	Hybrid Software Analysis.
	Software Visualization.
\end{resumo}
            % Resumo em língua inglesa

% Lista de Figuras
%
% Este arquivo não necessita de ser editado. A lista é gerada automaticamente.
%

\pdfbookmark[0]{\listfigurename}{lof}
\listoffigures*
\cleardoublepage
     % Lista de Figuras
% 
% Lista de Tabelas
%
% Este arquivo não necessita de ser editado. A lista é gerada automaticamente.
%

\pdfbookmark[0]{\listtablename}{lot}
\listoftables*
\cleardoublepage
      % Lista de Tabelas
% 
% Lista de Quadros
%
% Este arquivo não necessita de ser editado. A lista é gerada automaticamente.
%

\pdfbookmark[0]{\listofquadrosname}{loq}
\listofquadros*
\cleardoublepage
      % Lista de Quadros
\include{./1....pre-textual...elements/1.11....list...of...algorithms}  % Lista de Algoritmos
% 
% Lista de Siglas
%
% Edite a lista acima para definir "todos" os acrônimos e siglas utilizados
% neste trabalho
%

\begin{siglas}
    \item[ABNT] Associação Brasileira de Normas Técnicas
    \item[DECOM] Departamento de Computação
\end{siglas}
    % Lista de Abreviaturas e Siglas
% 
% Lista de Símbolos
%
% Edite a lista acima para definir "todos" os símbolos utilizados neste
% trabalho.
%

\begin{simbolos}
    \item[$ \Gamma $] Letra grega Gama
    \item[$ \lambda $] Comprimento de onda
    \item[$ \in $] Pertence
\end{simbolos}
     % Lista de Símbolos

% Sumário
%
% Este arquivo não necessita de ser editado. O sumário é gerado automaticamente.
%

\tableofcontents*
\cleardoublepage
                 % Sumário

\textual

% Introdução
%

\chapter{Introdução}
\label{chap:Introdução}

A Engenharia de Software é uma área incipiente do conhecimento, com menos
de um século de história. Consiste em uma disciplina que trata de questões
políticas, culturais, técnicas e tecnológicas dentro de organizações.
Incorporou, primeiramente, interesse por processos, modelos de processos,
planejamento e gestão de projetos; e, posteriormente, pela qualidade de
sistemas de software \cite{Wazlawick2013:Engenharia}.

\citeonline{Dijkstra:1972:HumbleProgrammer} tratou do fenômeno da crise do
software, onde relatou dificuldade de agentes interessados na Engenharia de
Software em cumprir prazos, cumprir orçamentos, entregar produtos de boa
qualidade e, por fim, desenvolver, manter e evoluir sistemas de software.

As atividades de desenvolvimento, evolução e manutenção de software têm natureza
cognitiva \cite{Letovsky1987:Cognitive}, uma vez que se faz necessário aplicar
conhecimentos --- técnicos, tecnológicos, entre outros --- para mapear a
concepção de um sistema de software no sistema de fato
\cite{Brooks1983:TheoryComprehension}. Outra forma de se dizer isto é que 
existem camadas intermediárias entre desejos e necessidades e os sistesmas
de software que possivelmente as sanam (Figura \ref{fig:CamadasAbstraçãoSoftware}).

\begin{figure}[!htb]
    \centering
    \caption{Camadas entre desejos e necessidades e os sistemas de software}
    \includegraphics[width=1\textwidth]{../shared files/figures/software abstraction layers/main.pdf}
    \fonte{do autor, 2017}
    \label{fig:CamadasAbstraçãoSoftware}
\end{figure}

A quinta Lei de Lehman \cite{Lehman1980:Laws}, a lei da conservação da
familiaridade: complexidade percebida, determina que a taxa de crescimento de
um sistema é limitada pela quantidade de informação absorvida coletiva e
individualmente \cite{Wazlawick2013:Engenharia}.

De acordo com \citeonline[p.~94]{Crockford2008:JavaScriptGoodParts}:

\begin{citacao}
\textit{Computer programs are the most complex things that humans make. Programs
are made up of a huge number of parts, expressed as functions, statements, and
expressions that are arranged in sequences that must be virtually free of error.
The runtime behavior has little resemblance to the program that implements it.
Software is usually expected to be modified over the course of its productive
life. The process of converting one correct program into a different correct
program is extremely challenging.}
\end{citacao}

Fica caracterizado o problema que é conhecer e alterar as propriedades dos
sistemas de software corretamente; portanto.

Diversas abordagens surgiram ao longo dos anos para tratar tais questões, de
modo que ferramentas, metodologias, processos e modelos foram propostos por
praticantes e estudiosos. Formou-se então, entre outras campos de pesquisa nas
áreas de Cognição, Compreensão e Visualização de Software.

O campo de pesquisa referente à Visualização de Software recebeu considerável
atenção de pesquisadores nos últimos 30 anos, onde uma parcela das ferramentas
desenvolvidas foram baseadas em grafos \cite{Storey2006:Theories} e em grafos
hierárquicos \cite{Jahnke2002:Visualizations}.

Grafos hierárquicos atendem ao requisito funcional de abstração, comum entre os
trabalhos da área. Tal requisito considera que pessoas interessadas trabalham
com diferentes níveis de abstração, de forma a reduzir a quantidade de
informação através do agrupamento de elementos \cite{Kienle2007:Requirements}.

Redes complexas diferem dos sistemas dinâmicos tradicionais, pois não há a
necessidade de componentes se ligarem uns aos outros de forma homogênea e
regular \cite{Sayama:2015:Complex}.
\citeonline{Weifeng:2011:Complex} cita a aplicação da análise de redes complexas
em sistemas de software para sua caracterização, medição e modelagem de
crescimento, bem como aplicações nas práticas de Engenharia de Software.

%Os trabalhos de \citeonline{Barabasi:19999:EmergenceScaling} e \citeonline{Strogatz:1998:Networks}
%impulsionaram a teoria de redes complexas às mais diversas áreas do
%conhecimento, tal como Biologia, Medicina, Ciências Sociais, Engenharias ---
%Engenharia de Software inclusa ---, entre tantas outras
%\cite{Barabasi:2003:Linked}, \cite{Borner:2007:Network}, \cite{Barabasi:2016:Network}.

A estrutura de um software em tempo de execução difere de sua estrutura
estática \cite{Crockford2008:JavaScriptGoodParts}.
Nas palavras de \citeonline[p.~22]{Gamma1995:Design}:

\begin{citacao}
\textit{An object-oriented program's run-time structure often bears little
resemblance to its code structure. The code structure is frozen at compile-time;
it consists of classes in fixed inheritance relationships. A program's run-time
structure consists of rapidly changing networks of communicating objects. In
fact, the two structures are largely independent. Trying to understand one from
the other is like trying to understand the dynamism of living ecosystems from
the static taxonomy of plants and animals, and vice versa.}
\end{citacao}

A análise da estrutura dinâmica de um software permite o estudo de seu
comportamento e de seu fluxo de dados, já que certas propriedades desses
sistemas de software somente são conhecidas durante sua execução. A análise da
estrutura estática de um software, por sua vez, permite estudar representações
do código fonte e de certas relações entre seus componentes.

Relações entre as áreas do conhecimento da Engenharia de Software (ES),
Compreensão de Software (CS) e Visualização de Software (VS), bem como o emprego
de grafos hierárquicos e redes complexas (RC) e a análise da estruturas
estáticas (EE) e dinâmicas (ED) de sistesmas de software estão ilustradas na
Figura \ref{fig:RelaçãoConhecimentoTrabalho}.

\begin{figure}[!htb]
    \centering
    \caption{Relações entre as áreas do conhecimento e o trabalho da dissertação}
    \includegraphics[width=1\textwidth]{../shared files/figures/research area/main.pdf}
    \fonte{do autor, 2017}
    \label{fig:RelaçãoConhecimentoTrabalho}
\end{figure}

\section{Objetivos}
\label{sec:Objetivos}

Esta dissertação tem como objetivo geral propor um modelo baseado em grafos
hierárquicos para representar a estrutura de execução de sistemas de software
que exponha suas propriedades.
Elementos da rede são componentes de tais sistemas em determinada granularidade,
homogênea ou não.

\subsection{Perguntas de Pesquisa}
\label{subsec:PerguntasPesquisa}

Conduziremos o projeto de pesquisa de modo a responder às seguintes perguntas de
pesquisa:

\noindent
\texttt{\textit{PP \#1 -- }}
Que propriedades de sistemas de software o modelo é capaz de expor?

\noindent
\texttt{\textit{PP \#2 -- }}
Quais são as aplicações e limitações do modelo?

\subsection{Objetivos Específicos}
\label{subsec:ObjetivosEspecíficos}

Os objetivos específicos desta dissertação serão separados em dois grupos, que
tratarão dos objetivos específicos referentes à proposta do modelo e sua
implementação, haja vista distinção entre o campo teórico e prático.

Quanto aos objetivos específicos referentes à proposta do modelo, determinar
dados de entrada, transformações sobre tais dados e construção do modelo,
operações sobre o modelo, análise sobre o modelo, resultados e analise dos
resultados.

Quanto aos objetivos específicos referentes à implementação do modelo,
determinar desafios técnicos e tecnológicos, requisitos da solução, aplicações e
limitações.

\section{Organização do trabalho}
\label{sec:OrganizaçãoTrabalho}

No Capítulo 2 apresentaremos os trabalhos relacionados: ferramentas, metodologia
e modelos utilizados para apoiar profissionais nas atividades de 
desenvolvimento, manutenção e evolução de software. Serão apresentadas as
ferramentas comumente utilizadas --- como \textit{loggers}, depuradores e
\textit{profilers} ---, trabalhos inseridos nas áreas de Cognição, Compreensão
e Visualização de Software, bem como trabalhos onde a análise dos sistemas de
software é feita por meio da Teoria de Redes Complexas.

No Capítulo 3 abordaremos a fundamentação teórica, onde constará o conhecimento
base para a compreensão deste trabalho. Discutiremos brevemente a história da
Engenharia de Software. Em seguida, visitaremos alguns conceitos da Teoria de
Grafos e Redes. Também trataremos da cognição, compreensão e visualização de
software.

No Capítulo 4 discutiremos quais são os requisitos do modelo proposto, bem como
quais critérios utilizar para verificá-lo e validá-lo. Explanaremos o modelo de
estrutura de software em execução através de redes hierárquicas de forma
discursiva e formal.

No Capítulo 5 analisaremos e discutiremos o modelo proposto quanto às suas
aplicações, limitações e demais características. Discutiremos ainda aspectos
sobre possível implementação do método em uma ferramenta de visualização de
software.

No Capítulo 6 apresentaremos as conclusões desta dissertação, enunciaremos
possibilidades de trabalhos futuros e as considerações finais.
                % Introdução

% Trabalhos Relacionados
%

\chapter{Trabalhos Relacionados}
\label{chap:RelatedWork}

Neste capítulo serão apresentados modelos, ferramentas e metodologia utilizados
para apoiar profissionais nas atividades de desenvolvimento, manutenção e
evolução de software. Serão apresentadas as ferramentas comumente utilizadas ---
como \textit{loggers}, depuradores e \textit{profilers} ---, trabalhos inseridos
nas áreas de Cognição, Compreensão e Visualização de Software, bem como
trabalhos onde a análise dos sistemas de software é feita por meio da Teoria de
Redes Complexas.

\section{\textit{Loggers}, Depuradores e \textit{Profilers}}
\label{sec:Loggers}

\textit{Loggers} são ferramentas que monitoram a execução de sistemas de
software através de registros ou entradas de \textit{log}.
Os registros podem estar dispostos em um formato específico ---
\textit{extensible markup language} (XML), \textit{JavaScript Object Notation} (JSON),
entre outros --- ou constar como texto informativo.
Podem ainda estar associados a níveis distintos --- \textit{information},
\textit{warning}, \textit{error}, entre outros.
As informações podem ser veiculadas a diferentes tipos de mídia, como arquivos
texto, arquivos binários, bases de dados relacionais, não relacionais ou
armazenadas em memória.

A análise de um conjunto de registros usualmente se dá por cálculos estatísticos
e por meio de mecanismos de buscas convencionais na mídia nos quais eles foram
veiculados.
Tais ferramentas possuem diferentes métodos de entradas. Seus recursos podem ser
acessados através de um protocolo específico, de uma
\textit{application program interface} (API) ou de funções e métodos.
São exemplos de \textit{loggers}:
SmartInspect\footnote{Disponível em: \href{http://www.gurock.com/smartinspect/}{http://www.gurock.com/smartinspect/}. Acessado em 18/05/2017.},
Logentries\footnote{Disponível em: \href{http://logentries.com/}{http://logentries.com/}. Acessado em 18/05/2017.} e
Logentries\footnote{Disponível em: \href{http://sentry.io/welcome/}{http://sentry.io/welcome/}. Acessado em 18/05/2017.}.

%\section{Depuradores}
%\label{sec:Debuggers}

Depuradores são ferramentas que possibilitam avaliar a execução de sistemas de
software. Tais ferramentas permitem definir pontos de parada ---
\textit{breakpoints}, \textit{watchpoints} e \textit{catchpoints} ---,
condicionais ou não, nos quais o depurador pausa a execução. Uma vez pausada a
execução do sistema, pode-se executar até o próximo ponto de parada, próxima
expressão, ou adentrar uma definição --- como um método ou função. Permitem ainda
avaliar determinadas expressões e monitorar certos valores de variáveis do
sistema.

Alguns depuradores implementam funcionalidades que permitem a depuração de:
sistemas já em execução; software em execução em máquina remota; programas
\textit{multithread}. Outras funcionalidades e técnicas de depuração são
disponibilizadas por depuradores e desenvolvidas por pesquisadores.
\citeonline{Engblom:ReverseDebuggin:2012} revisou o estado da arte referente à
depuração reversa, que consiste em interromper a execução do software ao se
constatar uma falha e desfazer o histórico de execuções para se avaliar o que a
causou.
São exemplos de depuradores \textit{The GNU Project Debugger} (GDB)
\footnote{Disponível em: \href{http://www.gnu.org/s/gdb/}{http://www.gnu.org/s/gdb/}. Acessado em 18/05/2017.} e 
\textit{The Java Debugger} (JDB)
\footnote{Verifique: \href{http://www.tutorialspoint.com/jdb/}{http://www.tutorialspoint.com/jdb/}. Acessado em 18/05/2017.}.

%\section{\textit{Profilers}}
%\label{sec:Profilers}

\textit{Profilers} são ferramentar utilizadas para extrair dados da execução de
sistemas de software através de técnicas de análise estática e dinâmica.
\citeonline{Thiel:2006:Profiling} analisa 8 ferramentas e discorre sobre
instrumentação em tempo de compilação, ferramentas de amostragem, instrumentação
através de contadores de hardware, instrumentação binária, dentre outras
técnicas.
São exemplos de ferramentas de \textit{profiling} Gprof\footnote{Verifique: \href{https://sourceware.org/binutils/docs/gprof/}{https://sourceware.org/binutils/docs/gprof/}. Acessado em 20/05/2017}
e Dtrace\footnote{Verifique: \href{http://dtrace.org/blogs/about/}{http://dtrace.org/blogs/about/}. Acessado em 20/05/2017}.

\citeonline{Henderson:2017:Software} relatou as principais práticas de
Engenharia de Software praticadas pela empresa Google\footnote{Disponível em:
\href{http://www.google.com/}{http://www.google.com/}. Acessado em 20/05/2012}.
Na seção referente às ferramentas de depuração e \textit{profiling}, Henderson
constata que os servidores da empresa possuem bibliotecas que disponibilizam
determindas ferramentas de análise. No caso de falha, informações da execução do
serviço --- amostradas, em alguns casos --- são salvas em arquivos de
\textit{log}. Menciona ainda sobre uma interface web para depurar 
\textit{remote procedure calls} (RCPs), alterar argumentos de linha de comando,
avaliar uso de recursos computacionais, \textit{profiling}, entre outras coisas.
Discorre brevemente, por fim, da facilidade de se depurar sistemas de software
na empresa e o quão raras são as ocasiões em que se mostra necessário utilizar
um depurador convencional como o GDB.

O modelo proposto neste trabalho irá registrar trilhas de execução em
\textit{log}. A instrumentação do sistema de software estudado pode ser
contínua ou desabilitada --- até mesmo amostrada. A integração do modelo com
a execução do sistema em tempo real não foi considerada, embora tecnicamente
possível. Isso viabilizaria, por exemplo, uma técnica de depuração através da
rede do software.

\section{Cognição, Compreensão e Visualização de Software}
\label{sec:SoftwareCognition}

\citeonline{Pacione:2004:Software} teorizou um modelo com três dimensões de
abstração.
A primeira dimensão trata da granularidade dos elementos de tal
modelo: em um nível de menor granularidade são consideradas as instruções e
expressões; componentes e pacotes representam um nível maior de granularidade.
A segunda dimensão trata das diversas facetas pelos quais os componentes podem
ser avaliados, como por exemplo a análise da estrutura do software, de seu 
comportamento ou do fluxo de dados.
A terceira dimensão trata do tipo de análise empregada, se estática ou dinâmica.
\citeonline{Pacione:2004:NovelModel} criou um modelo com base no modelo
teorizado por \citeonline{Pacione:2004:Software}.

No modelo proposto neste trabalho a granularidade selecionada para análise não
necessariamente é homogênea: enquanto parte do sistema é considerada em nível
maior granularidade, outras partes da rede podem ser expostas de forma
detalhada. Cada componente da rede remete a um ou mais elementos da estrutura
estática do software, enquanto que as ligações revelam as interações entre esses
componentes durante a execução do sistema. O fluxo de dados pode ser
parcialmente observado ao passo em que essas interações ocorrem, uma vez que o
modelo, a priori, não provê mecanismos de monitorar valores.

\citeonline{Hendrix:2002:ControlStructureDiagrams} propos diagramas de
estruturas de controle renderizados juntamente com o código-fonte no editor de
texto e validou a eficácia de tais diagramas através de dois experimentos, com
38 e 50 estudantes.
Os estudantes foram separados em dois grupos --- controle e experimental --- com
o mesmo número de indivíduos, para cada experimento.
A análise estatística rejeitou fortemente a hipótese nula do estudo que
considera não relevante que diagramas de estruturas de controle afetam
positivamente a performance dos entrevistados em responderem perguntas.

O destaque visual de certas informações pode auxiliar na compreensão da rede.

%\section{Visualização de Software}
%\label{sec:SoftwareVisualization}

%\section{Redes Complexas}
%\label{sec:ComplexNetworks}

\section{Análise de Software por Redes Complexas}
\label{sec:SoftwareComplexNetworks}

\citeonline{Valverde:2002:Complex} foram pioneiros ao introduzir a Teoria de
Redes Complexas à análise de sistemas de software, pois abstrairam um diagrama
de classes e modelaram uma rede com ligações não direcionadas entre classes e
métodos. O trabalho analisou 2 sistemas de software e constatou que ambos
apresentam propriedades de redes de mundo pequeno e de redes livre de escala.

\citeonline{Myers:2003:Complex} argumentou que a natureza da relação entre
componentes altera o fluxo de dados e é, portanto, significante, de modo que a
rede modelada deve conter ligações direcionadas entre seus nós. Este trabalho
estudou 3 sistemas orientados a objetos e outros 3 sistemas procedurais através
de grafos de colaboração obtidos por análise estática. Todos os sistemas de
software apresentaram características de redes de mundo pequeno e livres de
escala, despeito de suas diferenças em aspectos tecnológicos e metodológicos.

\citeonline{Valverde:2003:Complex} consideraram uma base de diagramas de classes
de 23 sistemas de software escritos em Java e C++. Modelaram redes com ligações
direcionadas a partir de tal base e observaram que além de tais redes
apresentarem estrutura de redes de mundo pequeno e redes livre de escala,
compartilham modularidade e características hierárquicas.

Certos autores estudaram componentes de sistemas de software em diferentes
níveis de granularidade.

\citeonline{Moura:2003:Signatures} analisou rede formada por cabeçalhos de
sistemas escritos em \texttt{C} e \texttt{C++} e verificou que tais redes
possuem características de rede de mundo pequeno e livre de escala.
Apontam ainda que o crescimento do sistema é o processo que faz com que a rede
seja livre de escala --- processo de ligação preferencial, conforme destacado
por \citeonline{Barabasi:1999:EmergenceScaling} ---, enquanto que o processo de
manutenção perfectiva é responsável pela configuração da rede de mundo pequeno.

\citeonline{LaBelle:2004:PackageNetworks} estudaram a rede de dependências entre
pacotes dos repositórios Debian GNU/Linux e FreeBSD Ports Collection e
identificou características de rede de mundo pequeno e livre de escala.

% EVOLUÇÃO

% METRICAS

% POO & POWERLAW

% TRABALHOS RECENTES
              % Trabalhos Relacionados
% \include{./2....textual...elements/2.03....theoretical...foundation}    % Fundamentação Teórica

% Metodologia
%

\chapter{Metodologia}
\label{chap:Methodoly}

\section{Modelo Proposto}
\label{sec:PropusedModel}

\subsection{Coleta e Tratamento de Dados}
\label{subsec:DataAquisition}

\subsection{Construção do Modelo}
\label{subsec:ModelConstruction}

\subsection{Operações sobre o Modelo}
\label{subsec:ModelOperations}

\subsection{Análise sobre o Modelo}
\label{subsec:ModelAnalysis}

\section{Implementação do Modelo}
\label{ModelImplementation}

\subsection{Desafios Técnicos e Tecnológicos}
\label{TechnicalChallenges}

\subsection{Requisitos da Solução}
\label{Requirements}
                 % Metodologia

% Resultados
%

\chapter{Resultados Preliminares}
\label{Chapter:Results}

Neste primeiro estudo de caso preliminar, avaliamos o sistema de software
\texttt{gulp}\footnote{Disponível em: \href{http://gulpjs.com}. Acessado em 22/07/2017.},
cuja estrutura de execução conta com 99 linhas de código, 3 arquivos fontes e 5
funções.
Este sistema foi escolhido por que foi a menor aplicação encontrada no trabalho
de \citeonline{silva2017identifying}, que selecionou os 1000 repositórios mais
populares de aplicações JavaScript de acordo com seu número de estrelas.
Tal seleção ocorreu em 2015.

O \texttt{gulp} é um \textit{task runner} JavaScript que permite automatizar tarefas
rotineiras do fluxo de desenvolvimento, tais quais construir aplicações,
executar testes unitários, entre outras.

Sua estrutura simplificada e $\mathcal{M}_{\texttt{gulp}}$
--- mapa da estrutura de execução do sistema de software \texttt{gulp}, com componentes
topológicos --- se encontram representados nas Figuras
\ref{Figure:GulpStructure} e \ref{Figure:GulpMap}, respectivamente.

\begin{figure}[!htb]
    \centering
    \caption{Estrutura simplificada para a aplicação \texttt{gulp}}
    \includegraphics[scale=1.2]{../shared files/figures/2.05....results/01.1....gulp...structure/main.pdf}
    \fonte{do autor, 2017}
    \label{Figure:GulpStructure}
\end{figure}

\begin{figure}[!htb]
    \centering
    \caption{Mapa $\mathcal{M}_{\texttt{gulp}}$ da estrutura de execução do sistema de software \texttt{gulp}, com componentes topológicos}
    \includegraphics[scale=1.2]{../shared files/figures/2.05....results/01.2....gulp...map/main.pdf}
    \fonte{do autor, 2017}
    \label{Figure:GulpMap}
\end{figure}

A rede foi produzida por análise estática, uma vez que se trata de uma aplicação
hipotética de pequeno porte e o resultado de sua instrumentação é simples de se
obter.
Não utilizamos o processo de simulação de forças elásticas e elétrica proposto
neste trabalho, dado que a implementação da ferramenta de simulação ainda não
foi concluída.
Componentes externos referentes aos módulos diretamente utilizados pela
aplicação avaliada foram adicionados. Dependências indiretas foram omitidas.

O componente anônimo definido por \texttt{index.js} é uma função definida e
no escopo do componente \texttt{watch}.
Isto mostra que a rede --- como definida --- não tem capacidade de representar
este nível de detalhamento.
Evitamos representar exceções e peculiaridades para não pesar o modelo visual
dos mapas renderizados.

%Um problema que é preciso enfrentar sempre que se decide representar
%informações em uma rede é o de como isso afeta a clareza visual da rede.

O \texttt{gulp} provê funcionalidades das mais diversas. No entanto,
$\mathcal{M}_{\texttt{gulp}}$ é uma rede simples, com poucos componentes e 
ligações. Como pode $\mathcal{M}_{\texttt{gulp}}$ prover tantas funcionalidades?
Além de depender de uma série de módulos, \texttt{gulp} é extensível por
intermédio de \textit{plugins}, o que nos faz pensar na necessidade de estudar
não só sistemas de software, mas também a interação entre tais sistemas: análise
de um sistema de sistemas.
%Como pode uma rede tão pequena quanto $\mathcal{M}_{\texttt{gulp}}$ ? Ela não provê. 
                     % Resultados

% Conclusão
%

\chapter{Conclusão}
\label{Chapter:Conclusion}

Nesta dissertação propomos uma metodologia para modelar e renderizar a estrutura
de execução de sistemas de software através de redes hierárquicas, heterogêneas
e navegáveis em múltiplas escalas.
Ela é aplicável a redes homogêneas, no entanto.
Esta abordagem é agnóstica de linguagens de programação.

%A mudança de escala índice-série permite analisar componentes com alto nível
%hierárquico, que nos informa da estrutura do sistema em análise, ao passo em que
%exibe detalhes de sua estrutura de execução.

% referenciar
Segundo \citeonline{ezzati2017multi}, abordagens que utilizam de trilhas de
execução para análise de software geram grandes quantidades de dados, difíceis
de analisar e gerenciar. Extrair informações pertinentes destes dados não é
tarefa trivial.
$\mathcal{M}$ reduz a quantidade de informação extraída das trilhas de execução
através da aplicação dos operadores de redução espacial $\mathcal{S}$ e temporal
$\mathcal{T}$ e pelas operações de agrupamento e supressão de seus elementos.

A valoração dos componentes da rede considera seu nível hierárquico. A valoração
das ligações depende do nível hierárquico do elemento originatário e do elemento
destinatário. Valoriza componentes e ligações de de baixo nível hierárquico, com
o objetivo de expor detalhes da estrutura de execução, bem como desvaloriza
ligações entre componentes de níveis hierárquicos distintos, uma vez que a
ligação entre componentes próximos revelam relações de funcionamento e 
colaboração, no lugar da noção estrutural estabelecida pelas ligações de
componentes distantes. Esta técnica não considera o tipo de ligação entre
componentes da rede, no entanto.

Ainda se faz necessário implementar ferramentas para extrair e tratar as trilhas
de execução, bem como renderizar a rede. Foram considerados casos didáticos para
a aplicação dos operadores de redução espacial $\mathcal{S}$ e temporal
$\mathcal{T}$, bem como avaliado a ferramenta de automatização de tarefas de
desenvolvimento nomeada gulp.

\section{Trabalhos Futuros}
\label{sec:FutureWork}

Como possibilidades de trabalhos futuros, vislumbramos a aplicação da Teoria de
Redes para obter informações relativas às propriedades topológicas e métricas
das redes. São redes de mundo pequeno? Suas estruturas são livres de escala?
Qual é seu coeficiente de agrupamento? Quais são suas métricas de centralidade?
Essas são algumas da muitas perguntas que se pode responder.
Além de extrair relatórios informativos quanto da distribuição de ligações por
componente, relacionar tais métricas com as métricas estabelecidas na Engenharia
de Software.

Existe a possibilidade de extensão do modelo para que ele permita: comparar
diferentes versões de um sistema;
realizar gerência de conhecimento;
rastrear requisitos;
análisar vários aspectos de qualidade;
verificar cobertura de cóodigo;
engenharia de software colaborativa;
auxiliar a gerência de projetos;
e outras muitas possibilidades. 

É necessário que se faça um estudo detalhado referente aos parâmetros do modelo,
tanto na valorização de componetes e ligações, quanto na renderização das redes.
Esse estudo pode abordar ainda o uso de diferente séries numéricas para tal
valoração.
Outro ponto de estudo referente ao modelo é quanto aos métodos de posicionamento
inicial de componentes, uma vez que o método proposto neste trabalho não se
preocupa com minimizar a energia potencial do sistema, ou utilizar o espaço de
forma eficiente, mas apenas para evitar interpenetrações severas nas primeiras
iterações da simulação de forças.

%Algumas das propriedades topológicas das redes: se são redes de mundo pequeno e
%livres de escala, coeficiente de clusterização, e outras.
%Quanto às métricas extraídas das redes, pode-se saber a distribuição de ligações
%por componentes, relação entre ligações de entrada e de saída, coeficiente de
%clusterização, métricas diversas de centralidade, dentre outras mais. Pode-se
%ainda extrair relações entre as métricas estabelecidas na Engenharia de Software
%e na Teoria de Redes.

%Também deve indicar, se possível e/ou conveniente, como este trabalho pode ser
%estendido ou aprimorado.

\section{Considerações Finais}
\label{sec:FinalConsiderations}

O modelo proposto possui limitações. Não apenas por se fazer valer de técnicas
híbridas de análise de sistemas de software, mas porque sua representação não
captura certas características de uma realidade mais complexa. Ora porque as
operações de redução espacial e temporal retiram parte da informação presente
nas trilhas de execuções instrumentadas; ora porque o modelo --- como definido
--- não é capaz de representar certos cenários. Nosso estudo de caso mostrou
que o modelo representa uma função definida no escopo de outra função como se
definida em um arquivo.

Se retomarmos as perguntas de pesquisa realizadas no começo deste texto, podemos
notar que ainda não temos todas as repostas.
É possível afirmar positivamente às perguntas de pesquisa \texttt{\#1} e
\texttt{\#3} (parcialmente), uma vez que modelamos a rede tal como proposto e ela
se mostrou extensível ao passo em que muito do que descrevemos na metodologia
não fazia parte da proposta inicial de trabalho.
A pergunta de pesquisa \texttt{\#2} ainda permanece sem resposta, uma vez que o
único sistema avaliado até o momento é de pequeno porte e a renderização
simulada não foi efetuada.

Para finalizarmos este trabalho ainda se mostra necessário: fundamentar os
conceitos utilizados; implementar ferramenta para extrair trilhas de execução de
sistemas em análise; implementar ferramenta para renderizar redes de tais
sistemas; realizar estudos de caso em número considerável de aplicações; 
discutir resultados encontrados.

%\begin{itemize}
%	\item Fundamentar os conceitos utilizados;
%	\item Implementar ferramenta para extrair trilhas de execução de sistemas em
%	análise;
%	\item Implementar ferramenta para renderizar redes de tais sistemas;
%	\item Realizar estudos de caso em número considerável de aplicações;
%	\item Discutir resultados encontrados.
%\end{itemize}
                  % Conclusão

\postextual

% References
%
% Loads refbase.bib file and extract cited references. This file does not need
% to be edited.
%

\bibliography{./refbase}{}
\bibliographystyle{abntex2-alf} % Define o estilo ABNT para formatar a lista de referências
             % Referências

% Apêndices
%

\begin{apendicesenv}
\partapendices

% Operadores de Redução Espacial $\mathcal{S}$ e Temporal $\mathcal{T}$: 
\chapter{Casos Didáticos}
\label{Chapter:SpatialTemporalOperators}

\textit{Falta escrever.}

\end{apendicesenv}
               % Apêndices
% \include{./3....post-textual...elements/3.03....annex}                  % Anexos
% 
% Índice Remissivo
%
%
% Este comando gera automaticamente o índice remissivo para os termos definidos
% no corpo do documento.
%
% Este arquivo não necessita de ser editado.
%

\printindex
      % Índice Remissivo

\end{document}
