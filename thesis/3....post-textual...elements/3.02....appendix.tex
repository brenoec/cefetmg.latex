
% Apêndices
%

\begin{apendicesenv}
\partapendices

% Operadores de Redução Espacial $\mathcal{S}$ e Temporal $\mathcal{T}$: 
\chapter{Casos Didáticos}
\label{Chapter:SpatialTemporalOperators}

Este capítulo contém caso didático no qual aplicamos os operadores de redução
espacial e temporal em trilhas de execução do algoritmo fatorial recursivo e
obtemos o mapa da estrutura de sua execução.
Em seguida, consideramos um sistema hipotético de loja virtual no qual, além de
aplicar tais operadores para obter mapa do sistema, inserimos componentes
topológicos na rede e calculamos os valores do componentes e ligações.
Assumimos 1 para todos os parâmetros do modelo, com finalidade de simplificar os
cálculos e facilitar sua exibição em texto.

Os mapas da estrutura de execução dos casos exibidos neste capítulo não foram
simuladas através do modelo de forças elásticas e elétricas proposto nesta
dissertação, dado que a ferramenta de simulação ainda não foi concluída.

\section{Caso Didático: Algoritmo Fatorial Recursivo}

Podemos observar as possíveis trilhas de execução para o Algoritmo
\hyperref[Algorithm:RecursiveFactorial]{A.1} na Figura
\ref{Figure:RecursiveFatorialTraces}. Note que $x_1$ corresponde às trilhas de
execução para $n = 0$ e $n = 1$. A trilha para $n = 2$ é representada por $x_2$,
enquanto que $x_n$ representa trilhas para um $n$ qualquer.

\begin{algorithm}[!htb]
    \DontPrintSemicolon 

    \label{Algorithm:RecursiveFactorial}
    \caption{Fatorial recursivo}
    \BlankLine

    %\Entrada{número natural $n$}
    %\Saida{fatorial de $n$}
    %\BlankLine

    \SetKwFunction{FnFatorial}{\texttt{Fat}}
    \SetKwProg{Function}{Função}{:}{\textbf{fim}}
    \Function{\FnFatorial{$n$}}{

        \BlankLine
        \Se {$n = 0 \ \lor \  n = 1$} {
            \Retorna 1
        }

        \BlankLine
        \Senao {
            \Retorna $n \  \times \  $ \texttt{Fat($n - 1$)}
        }

        \BlankLine
        \Retorna não definido

        \BlankLine
    }
\end{algorithm}


Aplicamos então o operador de redução espacial $\mathcal{S}$ ao conjunto 
de trilhas de execução $X = \{\ x_1,\ x_2,\ \dots,\ x_n\ \}$ e obtemos o
conjunto $\dot{X} = \{\ \dot{x}_1,\ \dot{x}_2,\ \dots,\ \dot{x}_n\ \}$ (Figura
\ref{Figure:RecursiveFatorialSpatialReducedTraces}).
Esta operação une instâncias diferentes de um mesmo ponto de uma trilha de
execução. Observe que todos os elementos $x_i$ são iguais para $i \geq 2$.
Na sequência, aplicamos o operador de redução temporal $\mathcal{T}$ sobre o
conjunto de trilhas reduzidas espacialmente $\dot{X}$ e obtemos mapa
$\mathcal{M}$ para o algoritmo de fatorial recursivo (Figura
\ref{Figure:RecursiveFatorialCase}). Esta operação une instâncias de um mesmo
componente entre diferentes trilhas de execuções reduzidas espacialmente.

Este caso contém apenas um componente elementar ($i = 1$) com valor $v = 1$ e
laço com valor $e = 1$.

\begin{figure}[!htb]
    \centering
    \caption{Possíveis trilhas de execução para o Algoritmo \hyperref[Algorithm:RecursiveFactorial]{A.1}}
    \includegraphics[scale=1.2]{../shared files/figures/3.02....appendix/01.1....recursive...factorial/main.pdf}
    \fonte{do autor, 2017}
    \label{Figure:RecursiveFatorialTraces}
\end{figure}

\begin{figure}[!htb]
    \centering
    \caption{Operador de redução espacial $\mathcal{S}$ aplicado sobre elementos da Figura \ref{Figure:RecursiveFatorialTraces}}
    \includegraphics[scale=1.2]{../shared files/figures/3.02....appendix/01.2....recursive...factorial/main.pdf}
    \fonte{do autor, 2017}
    \label{Figure:RecursiveFatorialSpatialReducedTraces}
\end{figure}

\begin{figure}[!htb]
    \centering
    \caption{Operador de redução temporal $\mathcal{T}$ aplicado sobre elementos da Figura \ref{Figure:RecursiveFatorialSpatialReducedTraces}}
    \includegraphics[scale=1.2]{../shared files/figures/3.02....appendix/01.3....recursive...factorial/main.pdf}
    \fonte{do autor, 2017}
    \label{Figure:RecursiveFatorialCase}
\end{figure}

\section{Caso Didático: Sistema Hipotético de Loja Virtual}

Considere um sistema de loja virtual hipotético cuja estrutura --- árvore de
pastas, arquivos e funções --- e fluxo de execução --- setas pretas --- estão
ilustrados na Figura \ref{Figure:StrutctureExecutionFlowStore}.

Um usuário se identifica ao acessar o sistema e, em algum ponto da execução, a
função \texttt{GetSession} recupera sua última sessão.
Imagine que o sistema direciona o usuário para seu carrinho de compras caso este
contenha algum produto --- função \texttt{ShowCart}, que executa
\texttt{GetCart} ---, ou o direciona para avaliar produtos em destaque caso
contrário --- função \texttt{ShowFeaturedItems}.

\begin{figure}[!htb]
    \centering
    \caption{Extrutura e fluxo de execução de sistema hipotético de loja virtual}
    \includegraphics[scale=1.2]{../shared files/figures/3.02....appendix/02.1....system...example/main.pdf}
    \fonte{do autor, 2017}
    \label{Figure:StrutctureExecutionFlowStore}
\end{figure}

Possíveis trilhas de execução desse sistema estão representadas na Figura
\ref{Figure:ExecutionTracesStore}. Observe que a aplicação do operador de
redução espacial $\mathcal{S}$ não alteram os grafos que representam tais
trilhas, uma vez que não há ocorrências múltiplas de nenhum dos componentes.
A aplicação do operador de redução temporal $\mathcal{T}$ resulta em mapa
$\mathcal{M}$ ainda sem componentes topológicos, ilustrado na Figura
\ref{Figure:StoreMap}.

É interessante ressaltar que a análise de $\mathcal{M}$ não nos permite saber
se as ligações do componente \texttt{Main} são chamadas em sequência, possíveis
em uma mesma trilha de execução; se correspondem a casos de condicionais,
fruto da redução temporal em diversas trilhas; ou se ali estão por motivos
outros.

\pagebreak

\begin{figure}[!htb]
    \centering
    \caption{Possíveis trilhas de execução de sistema hipotético de loja virtual}
    \includegraphics[scale=1.2]{../shared files/figures/3.02....appendix/02.2....system...example/main.pdf}
    \fonte{do autor, 2017}
    \label{Figure:ExecutionTracesStore}
\end{figure}

\begin{figure}[!htb]
    \centering
    \caption{$\mathcal{M}$ para sistema hipotético de loja virtual, sem componentes topológicos}
    \includegraphics[scale=1.2]{../shared files/figures/3.02....appendix/02.3....system...example/main.pdf}
    \fonte{do autor, 2017}
    \label{Figure:StoreMap}
\end{figure}

%\pagebreak

Uma representação de $\mathcal{M}$ com componentes topológicos consta na
Figura \ref{Figure:StoreTopologicalMap}. Podemos notar a agregação dos
componentes topológicos \textit{pasta} \texttt{Users} e \textit{arquivo}
\texttt{Session} em um único componente, uma vez que eles constituem uma cadeia
linear de elementos topológicos, vide Figura \ref{Figure:StrutctureExecutionFlowStore}.

%O método utilizado para atribuir posicionamento inicial a cada um dos
%componentes de $\mathcal{M}$ em grade se encontra ilustrado na Figura
%\ref{Figure:StoreInitialPositioning}, conforme processo descrito no Capítulo 
%\ref{Chapter:Methodology}

\begin{figure}[!htb]
    \centering
    \caption{$\mathcal{M}$ para sistema hipotético de loja virtual, com componentes topológicos}
    \includegraphics[scale=1.2]{../shared files/figures/3.02....appendix/02.4....system...example/main.pdf}
    \fonte{do autor, 2017}
    \label{Figure:StoreTopologicalMap}
\end{figure}

%\begin{figure}[!htb]
%    \centering
%    \caption{Posicionamento inicial dos componentes em grade}
%    \includegraphics[scale=1.2]{../shared files/figures/3.02....appendix/02.5....system...example/main.pdf}
%    \fonte{do autor, 2017}
%    \label{Figure:StoreInitialPositioning}
%\end{figure}

Índices e Valores de componentes, bem como os valores de ligações, constam na
Tabela \ref{Table:StoreMapValues}. Espaços em branco tem valor 0. Observe como
que a matriz de adjacência de $\mathcal{M}$ é esparsa e como que os pesos das
ligações de \texttt{App} revelam maior coesão, do ponto de vista da estrutura
desta rede, entre \texttt{App} e \texttt{Store}.

\begin{landscape}
\begin{table}[!htb]
    \centering
    \caption{Valores para mapa $\mathcal{M}$ de sistema hipotético de loja virtual. \label{Table:StoreMapValues}}
    \begin{tabular}{r|C{1.2cm}C{1.2cm}C{1.2cm}C{1.2cm}C{1.2cm}C{1.2cm}C{1.2cm}C{1.2cm}C{1.2cm}C{1.2cm}C{1.2cm}}
        \toprule
            & \rotatebox[origin=c]{70}{\texttt{App}}
            & \rotatebox[origin=c]{70}{\texttt{Store}}
            & \rotatebox[origin=c]{70}{\texttt{Cart}}
            & \rotatebox[origin=c]{70}{\texttt{ShowCart}}
            & \rotatebox[origin=c]{70}{\texttt{GetCart}}
            & \rotatebox[origin=c]{70}{\texttt{FeaturedItems}}
            & \rotatebox[origin=c]{70}{\texttt{ShowFeaturedItems}}
            & \rotatebox[origin=c]{70}{\texttt{Users/Session}}
            & \rotatebox[origin=c]{70}{\texttt{GetSession}}
            & \rotatebox[origin=c]{70}{\texttt{Main}}
            & \rotatebox[origin=c]{70}{\texttt{Main}} \\
        \midrule
            Índices & 5 & 3 & 2 & 1 & 1 & 1 & 1 & 1 & 1 & 1 & 1 \\
            Valores & 8 & 3 & 2 & 1 & 1 & 1 & 1 & 1 & 1 & 1 & 1 \\
        \midrule
            \texttt{App}                &  & $1/48$ &  &  &  &  &  & $1/64$ &  & $1/64$ &  \\
            \texttt{Store}              &  &  & $1/6$ &  &  & $1/9$ &  &  &  &  &  \\
            \texttt{Cart}               &  &  &  & $1/4$ & $1/4$ &  &  &  &  &  &  \\
            \texttt{ShowCart}           &  &  &  &  & 1 &  &  &  &  &  &  \\
            \texttt{GetCart}            &  &  &  &  &  &  &  &  &  &  &  \\
            \texttt{FeaturedItems}      &  &  &  &  &  &  & 1 &  &  &  &  \\
            \texttt{ShowFeaturedItems}  &  &  &  &  &  &  &  &  &  &  &  \\
            \texttt{Users/Session}      &  &  &  &  &  &  &  &  & 1 &  &  \\
            \texttt{GetSession}         &  &  &  &  &  &  &  &  &  &  &  \\
            \texttt{Main}               &  &  &  &  &  &  &  &  &  &  & 1 \\
            \texttt{Main}               &  &  &  & 1 &  &  & 1 &  & 1 &  &  \\
        \bottomrule
    \end{tabular}
\end{table}
\end{landscape}

\end{apendicesenv}
